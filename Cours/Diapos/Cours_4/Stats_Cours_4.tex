%\documentclass[ignorenonframetext, compress, 9pt, xcolor=svgnames]{beamer} 
\documentclass[notes, ignorenonframetext, compress, 11pt, xcolor=svgnames, aspectratio=169]{beamer} 
\usepackage{pgfpages}
\usepackage{pdfpages}
% These slides also contain speaker notes. You can print just the slides,
% just the notes, or both, depending on the setting below. Comment out the want
% you want.
\setbeameroption{hide notes} % Only slide
%\setbeameroption{show only notes} % Only notes
%\setbeameroption{show notes on second screen=right} % Both
\usepackage{amsmath}
\usepackage{amsfonts}
\usepackage{amssymb}
\setbeamercolor{frametitle}{fg=MidnightBlue}

\setbeamercolor{sectionpage title}{bg=MidnightBlue}
\setbeamertemplate{frametitle}[default][center]
%\setbeamertemplate{frametitle}{\color{MidnightBlue}\centering\bfseries\insertframetitle\par\vskip-6pt}
\setbeamerfont{frametitle}{series=\bfseries}
\setbeamerfont{title}{series=\bfseries}
\setbeamerfont{sectionpage}{series=\bfseries}
%\setbeamercolor{section in head/foot}{bg=MidnightBlueBlue}
%\setbeamercolor{author in head/foot}{bg=DarkBlue}
\setbeamercolor{author in head/foot}{fg=MidnightBlue}
%\setbeamercolor{title in head/foot}{bg=White}
\setbeamercolor{title in head/foot}{fg=MidnightBlue}
\setbeamercolor{title}{fg=MidnightBlue}
%\setbeamercolor{date in head/foot}{fg=Brown}
%\setbeamercolor{alerted text}{fg=DarkBlue}
%\usecolortheme[named=DarkBlue]{structure} 
%\usepackage{bbm}
%\usepackage{bbold}
\usepackage{eurosym}
\usepackage{graphicx}
%\usepackage{epstopdf}
\usepackage{hyperref}
\hypersetup{
  colorlinks   = true, %Colours links instead of ugly boxes
  urlcolor     = gray, %Colour for external hyperlinks
  linkcolor    = MidnightBlue, %Colour of internal links
  citecolor   = DarkRed %Colour of citations
}
\usepackage{multirow}
\usepackage{xspace}
\usepackage{listings}
\usepackage{natbib}
%\usepackage[sort&compress,comma,super]{natbib}
\def\newblock{} % To avoid a compilation error about a function \newblock undefined
\usepackage{bibentry}
\usepackage{booktabs}
\usepackage{dcolumn}
\usepackage[greek,frenchb]{babel}
\usepackage[babel=true,kerning=true]{microtype}
\usepackage[utf8]{inputenc}
\usepackage[T1]{fontenc}
\usepackage{natbib}
\renewcommand{\cite}{\citet}
\usepackage{longtable}
\usepackage{eso-pic}

\usepackage{xcolor}
 \colorlet{linkequation}{DarkRed} 
 \newcommand*{\SavedEqref}{}
 \let\SavedEqref\eqref 
\renewcommand*{\eqref}[1]{%
\begingroup \hypersetup{
      linkcolor=linkequation,
linkbordercolor=linkequation, }%
\SavedEqref{#1}%
 \endgroup
}

\newcommand*{\refeq}[1]{%
 \begingroup
\hypersetup{ 
linkcolor=linkequation, 
linkbordercolor=linkequation,
}%
\ref{#1}%
 \endgroup
}

\setbeamertemplate{caption}[numbered]
\setbeamertemplate{theorem}[ams style]
\setbeamertemplate{theorems}[numbered]
%\usefonttheme{serif}
%\usecolortheme{beaver}
%\usetheme{Hannover}
%\usetheme{CambridgeUS}
%\usetheme{Madrid}
%\usecolortheme{whale}
%\usetheme{Warsaw}
%\usetheme{Luebeck}
%\usetheme{Montpellier}
%\usetheme{Berlin}
%\setbeamercolor{titlelike}{parent=structure}
%\setbeamertemplate{headline}[default]
%\setbeamertemplate{footline}[default]
%\setbeamertemplate{footline}[Malmoe]
%\setbeamercovered{transparent}
%\setbeamercovered{invisible}
%\usecolortheme{crane}
%\usecolortheme{dolphin}
%\usepackage{pxfonts}
%\usepackage{isomath}
%\usepackage{mathpazo}
%\usepackage{arev} %     (Arev/Vera Sans)
%\usepackage{eulervm} %_   (Euler Math)
%\usepackage{fixmath} %  (Computer Modern)
%\usepackage{hvmath} %_   (HV-Math/Helvetica)
%\usepackage{tmmath} %_   (TM-Math/Times)
%\usepackage{tgheros}
%\usepackage{cmbright}
%\usepackage{ccfonts} \usepackage[T1]{fontenc}
%\usepackage[garamond]{mathdesign}

%\usepackage{color}
%\usepackage{ulem}

%\usepackage[math]{kurier}
%\usepackage[no-math]{fontspec}
%\setmainfont{Fontin Sans}
%\setsansfont{Fontin Sans}
%\setbeamerfont{frametitle}{size=\LARGE,series=\bfseries}
%%%add 19022021
\usepackage{enumerate}    
\usepackage{dcolumn}
\usepackage{verbatim}
\newcolumntype{d}[0]{D{.}{.}{5}}
%\setbeamertemplate{note page}{\pagecolor{yellow!5}\insertnote}
%\usetikzlibrary{positioning}
%\usetikzlibrary{snakes}
%\usetikzlibrary{calc}
%\usetikzlibrary{arrows}
%\usetikzlibrary{decorations.markings}
%\usetikzlibrary{shapes.misc}
%\usetikzlibrary{matrix,shapes,arrows,fit,tikzmark}
%%%
% suppress navigation bar
\beamertemplatenavigationsymbolsempty
%\usetheme{bunsenMod}
%\setbeamercovered{transparent}
%\setbeamertemplate{items}[circle]
%\usecolortheme[named=CadetBlue]{structure}
%\usecolortheme[RGB={225,64,5}]{structure}
%\definecolor{burntRed}{RGB}{225,64,5}
%\setbeamercolor{alerted text}{fg=burntRed} 
%\usecolortheme[RGB={0,40,110}]{structure}
%\hypersetup{linkcolor=burntRed}
%\hypersetup{urlcolor=burntRed}
%\hypersetup{filecolor=burntRed}
%\hypersetup{citecolor=burntRed}

%\usetheme{bunsenMod}
%\setbeamercovered{transparent}
%\setbeamertemplate{items}[circle]
%\usecolortheme[named=CadetBlue]{structure}
%\usecolortheme[RGB={225,64,5}]{structure}
%\definecolor{burntRed}{RGB}{225,64,5}
%\setbeamercolor{alerted text}{fg=burntRed} 
%\usecolortheme[RGB={0,40,110}]{structure}
%\hypersetup{linkcolor=burntRed}
%\hypersetup{urlcolor=burntRed}
%\hypersetup{filecolor=burntRed}
%\hypersetup{citecolor=burntRed}

%\AtBeginSection[] % Do nothing for \section*
%{ \frame{\sectionpage} }
%\setbeamertemplate{frametitle continuation}{}
\newtheorem{lemme}{Lemme}[section]
%\newtheorem{remarque}{Remarque}
\newcommand{\argmax}{\operatornamewithlimits{arg\,max}}
\newcommand{\argmin}{\operatornamewithlimits{arg\,min}}
\def\inprobLOW{\rightarrow_p}
\def\inprobHIGH{\,{\buildrel p \over \rightarrow}\,} 
\def\inprob{\,{\inprobHIGH}\,} 
\def\indist{\,{\buildrel d \over \rightarrow}\,} 
\def\sima{\,{\buildrel a \over \sim}\,} 
\def\F{\mathbb{F}}
\def\R{\mathbb{R}}
\def\N{\mathbb{N}}
\newcommand{\gmatrix}[1]{\begin{pmatrix} {#1}_{11} & \cdots &
    {#1}_{1n} \\ \vdots & \ddots & \vdots \\ {#1}_{m1} & \cdots &
    {#1}_{mn} \end{pmatrix}}
\newcommand{\iprod}[2]{\left\langle {#1} , {#2} \right\rangle}
\newcommand{\norm}[1]{\left\Vert {#1} \right\Vert}
\newcommand{\abs}[1]{\left\vert {#1} \right\vert}
\renewcommand{\det}{\mathrm{det}}
\newcommand{\rank}{\mathrm{rank}}
\newcommand{\spn}{\mathrm{span}}
\newcommand{\row}{\mathrm{Row}}
\newcommand{\col}{\mathrm{Col}}
\renewcommand{\dim}{\mathrm{dim}}
\newcommand{\prefeq}{\succeq}
\newcommand{\pref}{\succ}
\newcommand{\seq}[1]{\{{#1}_n \}_{n=1}^\infty }
\renewcommand{\to}{{\rightarrow}}
\renewcommand{\L}{{\mathcal{L}}}
\newcommand{\Er}{\mathrm{E}}
\renewcommand{\Pr}{\mathrm{P}}
%\newcommand{\Var}{\mathrm{Var}}
%\newcommand{\Cov}{\mathrm{Cov}}
%\newcommand{\corr}{\mathrm{Corr}}
%\newcommand{\Var}{\mathrm{Var}}
\newcommand{\bias}{\mathrm{Bias}}
\newcommand{\mse}{\mathrm{MSE}}
\providecommand{\Pred}{\mathcal{P}}
\providecommand{\plim}{\operatornamewithlimits{plim}}
\providecommand{\avg}{\frac{1}{n} \underset{i=1}{\overset{n}{\sum}}}
\providecommand{\sumin}{{\sum_{i=1}^n}}
\providecommand{\limp}{\overset{p}{\rightarrow}}
%\providecommand{\limp}{\underset{n \rightarrow \infty}{\overset{p}{\longrightarrow}}}
%\providecommand{\limp}{\underset{n \rightarrow \infty}{\overset{p}{\longrightarrow}}}
%\providecommand{\limp}{\overset{p}{\longrightarrow}}
%\providecommand{\limd}{\underset{n \rightarrow \infty}{\overset{d}{\longrightarrow}}}
\providecommand{\limd}{\overset{d}{\rightarrow}}
\providecommand{\limps}{\overset{p.s.}{\rightarrow}}
\providecommand{\limlp}{\overset{L^p}{\rightarrow}}
\def\independenT#1#2{\mathrel{\setbox0\hbox{$#1#2$}%
    \copy0\kern-\wd0\mkern4mu\box0}} 
\newcommand\indep{\protect\mathpalette{\protect\independenT}{\perp}}


\lstset{language=R}
\lstset{keywordstyle=\color[rgb]{0,0,1},                                        % keywords
        commentstyle=\color[rgb]{0.133,0.545,0.133},    % comments
        stringstyle=\color[rgb]{0.627,0.126,0.941}      % strings
}       
\lstset{
  showstringspaces=false,       % not emphasize spaces in strings 
  columns=fixed,
  numbersep=3mm, numbers=left, numberstyle=\tiny,       % number style
  frame=none,
  framexleftmargin=5mm, xleftmargin=5mm         % tweak margins
}
\makeatletter
%\setbeamertemplate{frametitle continuation}{\gdef\beamer@frametitle{}}
\setbeamertemplate{frametitle continuation}{\frametitle{}}
%\setbeamertemplate{frametitle continuation}{\insertcontinuationcount}
\makeatother

\theoremstyle{remark}
\newtheorem{interpretation}{Interprétation}
\newtheorem*{interpretation*}{Interprétation}

\theoremstyle{remark}
\newtheorem{remarque}{Remarque}%[section]
\newtheorem*{remarque*}{Remarque}
\usepackage[framemethod=TikZ]{mdframed} 
\usepackage{showexpl}
%\newtheorem{step}{Step}[section]
%\newtheorem{rem}{Comment}[section]
%\newtheorem{ex}{Example}[section]
%\newtheorem{hist}{History}[section]
%\newtheorem*{ex*}{Example}
\theoremstyle{plain}
\newtheorem{propriete}{Propri\'et\'e}
\renewcommand{\thepropriete}{P\arabic{propriete}}
%\theoremstyle{definition}
%\newtheorem{definition}{Définition}%[section]
\theoremstyle{remark}
\newtheorem{exemple}{Exemple}
\newtheorem*{exemple*}{Exemple}

\newtheorem{theoreme}{Théorème}
\newtheorem{proposition}{Proposition}
%\newtheorem{propriete}{Propri\'et\'e}
\newtheorem{corollaire}{Corollaire}
%\newtheorem{exemple}{Exemple}
\newtheorem{assumption}{Assumption}
\renewcommand{\theassumption}{A\arabic{assumption}}
\newtheorem{hypothese}{Hypothèse}
\renewcommand{\thehypothese}{H\arabic{hypothese}}
\theoremstyle{definition}

%\newtheorem{definitionx}{D\'efinition}%[section]
%\newenvironment{definition}
 %{\pushQED{\qed}\renewcommand{\qedsymbol}{$\triangle$}\definitionx}
 %{\popQED\enddefinitionx}

\newtheorem{condition}{Condition}
\renewcommand{\thecondition}{C\arabic{condition}}
%\newcommand{\Var}{\mathbb{V}}
%\newcommand{\Var}{\mathbf{Var}}
%\newcommand{\Exp}{\mathbf{E}}
%\providecommand{\Vr}{\mathrm{Var}}
%\renewcommand{\Er}{\mathbb{E}}
%\newcommand{\LP}{\mathcal{LP}}
%\providecommand{\Id}{\mathbf{I}}
%\providecommand{\Rang}{\mathrm{Rang}}
%\providecommand{\Trace}{\mathrm{Trace}}
%\newcommand{\Cov}{\mathbf{Cov}}
%\newcommand{\Cov}{\mathbb{C}\mathrm{ov}}
\providecommand{\Id}{\mathbf{I}}
\providecommand{\Ind}{\mathbf{1}}
\providecommand{\uvec}{\mathbf{1}}
\providecommand{\vecOnes}{\mathbf{1}}
\DeclareMathOperator{\indfun}{\mathbf{1}}
\DeclareMathOperator{\Exp}{E}
\DeclareMathOperator{\Expn}{\mathbb{E}_n}
\DeclareMathOperator{\Var}{Var}
\DeclareMathOperator{\Vr}{V}
\DeclareMathOperator{\Cov}{Cov}
\DeclareMathOperator{\corr}{corr}
\DeclareMathOperator{\perps}{\perp_s}
%\DeclareMathOperator{\Prob}{Pr}
\DeclareMathOperator{\Prob}{P}
\DeclareMathOperator{\prob}{p}
\DeclareMathOperator{\loss}{L}
\providecommand{\Corr}{\mathrm{Corr}}
\providecommand{\Diag}{\mathrm{Diag}}
\providecommand{\reg}{\mathrm{r}}
\providecommand{\Likelihood}{\mathrm{L}}
\renewcommand{\Pr}{{\mathbb{P}}}
\providecommand{\set}[1]{\left\{#1\right\}}
\providecommand{\uvec}{\mathbf{1}}
\providecommand{\Rang}{\mathrm{Rang}}
\providecommand{\Trace}{\mathrm{Trace}}
\providecommand{\Tr}{\mathrm{Tr}}
\providecommand{\CI}{\mathrm{CI}}
\providecommand{\asyvar}{\mathrm{AsyVar}}
\DeclareMathOperator{\Supp}{Supp}
\newcommand{\inputslide}[2]{{
    \usebackgroundtemplate{
     \includegraphics[page={#2},width=0.90\textwidth,keepaspectratio=true]
      %\includegraphics[page={#2},width=\paperwidth,keepaspectratio=true]
      {{#1}}}
    \frame[plain]{}
  }}
\newcommand\pperp{\perp\!\!\!\perp}
\newcommand\independent{\protect\mathpalette{\protect\independenT}{\perp}}
\def\independenT#1#2{\mathrel{\rlap{$#1#2$}\mkern2mu{#1#2}}}
\usepackage{bbm}
\providecommand{\Ind}{\mathbf{1}}
\newcommand{\sumjsi}{\underset{i<j}{{\sum}}}
\newcommand{\prodjsi}{\underset{i<j}{{\prod}}}
\newcommand{\sumisj}{\underset{j<i}{{\sum}}}
\newcommand{\prodisj}{\underset{j<i}{{\prod}}}
\newcommand{\sumobs}{\underset{i=1}{\overset{n}{\sum}}}
\newcommand{\sumi}{\underset{i=1}{\overset{n}{\sum}}}
\newcommand{\prodi}{\underset{i=1}{\overset{n}{\prod}}}
\newcommand{\prodobs}{\underset{i=1}{\overset{n}{\prod}}}
\newcommand{\simiid}{{\overset{i.i.d.}{\sim}}}
%\newcommand{\sumobs}{\sum_{i=1}^N}
%\newcommand{\prodobs}{\prod_{i=1}^N}
%\newcommand{\sumjsi}{\sum_{i<j}}
%\newcommand{\prodjsi}{\prod_{i<j}}
%\newcommand{\sumisj}{\sum_{j<i}}
%\newcommand{\prodisj}{\sum_{j<i}}

%\usepackage{appendixnumberbeamer}
\setbeamertemplate{footline}[frame number]
\setbeamertemplate{section in toc}[sections numbered]
\setbeamertemplate{subsection in toc}[subsections numbered]
\setbeamertemplate{subsubsection in toc}[subsubsections numbered]

%\makeatother
%\setbeamertemplate{footline}
%{
%    \leavevmode%
%    \hbox{%
%        \begin{beamercolorbox}[wd=.333333\paperwidth,ht=2.25ex,dp=1ex,center]{author in head/foot}%
%            \usebeamerfont{author in head/foot}\insertshortauthor
%        \end{beamercolorbox}%
%        \begin{beamercolorbox}[wd=.333333\paperwidth,ht=2.25ex,dp=1ex,center]{title in head/foot}%
%            \usebeamerfont{title in head/foot}\insertshorttitle
%        \end{beamercolorbox}%
%        \begin{beamercolorbox}[wd=.333333\paperwidth,ht=2.25ex,dp=1ex,right]{date in head/foot}%
%            \usebeamerfont{date in head/foot}\insertshortdate{}\hspace*{2em}
%            \insertframenumber{} / \inserttotalframenumber\hspace*{2ex} 
%        \end{beamercolorbox}}%
%       \vskip0pt%
 %   }
%   \makeatother
%\setbeamertemplate{navigation symbols}{}
\setbeamertemplate{itemize items}[ball]
%\setbeamertemplate{itemize items}{-}
%\newenvironment{wideitemize}{\itemize\addtolength{\itemsep}{10pt}}{\enditemize}
% \usepackage{eso-pic}
%\newcommand\AtPagemyUpperLeft[1]{\AtPageLowerLeft{%
%\put(\LenToUnit{0.9\paperwidth},\LenToUnit{0.9\paperheight}){#1}}}
%\AddToShipoutPictureFG{
%  \AtPagemyUpperLeft{{\includegraphics[width=1.1cm,keepaspectratio]{../logo-uga.png}}}
%}%
\def\figheight{3in}
\def\figwidth{4in}

%%Commands from Econometric Theory(Slides) by J. Stachurski.

\newcommand{\boldx}{ {\mathbf x} }
\newcommand{\boldu}{ {\mathbf u} }
\newcommand{\boldv}{ {\mathbf v} }
\newcommand{\boldw}{ {\mathbf w} }
\newcommand{\boldy}{ {\mathbf y} }
\newcommand{\boldb}{ {\mathbf b} }
\newcommand{\bolda}{ {\mathbf a} }
\newcommand{\boldc}{ {\mathbf c} }
\newcommand{\boldd}{ {\mathbf d} }
\newcommand{\boldi}{ {\mathbf i} }
\newcommand{\bolde}{ {\mathbf e} }
\newcommand{\boldp}{ {\mathbf p} }
\newcommand{\boldq}{ {\mathbf q} }
\newcommand{\bolds}{ {\mathbf s} }
\newcommand{\boldt}{ {\mathbf t} }
\newcommand{\boldz}{ {\mathbf z} }
\newcommand{\boldr}{ {\mathbf r} }

\newcommand{\boldzero}{ {\mathbf 0} }
\newcommand{\boldone}{ {\mathbf 1} }

\newcommand{\boldalpha}{ {\boldsymbol \alpha} }
\newcommand{\boldbeta}{ {\boldsymbol \beta} }
\newcommand{\boldgamma}{ {\boldsymbol \gamma} }
\newcommand{\boldtheta}{ {\boldsymbol \theta} }
\newcommand{\boldxi}{ {\boldsymbol \xi} }
\newcommand{\boldtau}{ {\boldsymbol \tau} }
\newcommand{\boldepsilon}{ {\boldsymbol \epsilon} }
\newcommand{\boldmu}{ {\boldsymbol \mu} }
\newcommand{\boldSigma}{ {\boldsymbol \Sigma} }
\newcommand{\boldOmega}{ {\boldsymbol \Omega} }
\newcommand{\boldPhi}{ {\boldsymbol \Phi} }
\newcommand{\boldLambda}{ {\boldsymbol \Lambda} }
\newcommand{\boldphi}{ {\boldsymbol \phi} }

\newcommand{\Sigmax}{ {\boldsymbol \Sigma_{\boldx}}}
\newcommand{\Sigmau}{ {\boldsymbol \Sigma_{\boldu}}}
\newcommand{\Sigmaxinv}{ {\boldsymbol \Sigma_{\boldx}^{-1}}}
\newcommand{\Sigmav}{ {\boldsymbol \Sigma_{\boldv \boldv}}}

\newcommand{\hboldx}{ \hat {\mathbf x} }
\newcommand{\hboldy}{ \hat {\mathbf y} }
\newcommand{\hboldb}{ \hat {\mathbf b} }
\newcommand{\hboldu}{ \hat {\mathbf u} }
\newcommand{\hboldtheta}{ \hat {\boldsymbol \theta} }
\newcommand{\hboldtau}{ \hat {\boldsymbol \tau} }
\newcommand{\hboldmu}{ \hat {\boldsymbol \mu} }
\newcommand{\hboldbeta}{ \hat {\boldsymbol \beta} }
\newcommand{\hboldgamma}{ \hat {\boldsymbol \gamma} }
\newcommand{\hboldSigma}{ \hat {\boldsymbol \Sigma} }

\newcommand{\boldA}{\mathbf A}
\newcommand{\boldB}{\mathbf B}
\newcommand{\boldC}{\mathbf C}
\newcommand{\boldD}{\mathbf D}
\newcommand{\boldI}{\mathbf I}
\newcommand{\boldL}{\mathbf L}
\newcommand{\boldM}{\mathbf M}
\newcommand{\boldP}{\mathbf P}
\newcommand{\boldQ}{\mathbf Q}
\newcommand{\boldR}{\mathbf R}
\newcommand{\boldX}{\mathbf X}
\newcommand{\boldU}{\mathbf U}
\newcommand{\boldV}{\mathbf V}
\newcommand{\boldW}{\mathbf W}
\newcommand{\boldY}{\mathbf Y}
\newcommand{\boldZ}{\mathbf Z}

\newcommand{\bSigmaX}{ {\boldsymbol \Sigma_{\hboldbeta}} }
\newcommand{\hbSigmaX}{ \mathbf{\hat \Sigma_{\hboldbeta}} }
\newcommand{\betahat}{\hat{\beta}}
\newcommand{\gammahat}{\hat{\gamma}}
\newcommand{\Uhat}{\hat{U}}
\newcommand{\Vhat}{\hat{V}}
\newcommand{\epsilonhat}{\hat{\epsilon}}
\newcommand{\sigmahat}{\hat{\sigma}}
\newcommand{\Sigmahat}{\hat{\Sigma}}
\newcommand{\Gammahat}{\hat{\Gamma}}

\newcommand{\RR}{\mathbbm R}
\newcommand{\CC}{\mathbbm C}
\newcommand{\NN}{\mathbbm N}
\newcommand{\PP}{\mathbbm P}
\newcommand{\EE}{\mathbbm E \nobreak\hspace{.1em}}
\newcommand{\EEP}{\mathbbm E_P \nobreak\hspace{.1em}}
\newcommand{\ZZ}{\mathbbm Z}
\newcommand{\QQ}{\mathbbm Q}


\newcommand{\XX}{\mathcal X}

\newcommand{\aA}{\mathcal A}
\newcommand{\fF}{\mathscr F}
\newcommand{\bB}{\mathscr B}
\newcommand{\iI}{\mathscr I}
\newcommand{\rR}{\mathscr R}
\newcommand{\dD}{\mathcal D}
\newcommand{\lL}{\mathcal L}
\newcommand{\llL}{\mathcal{H}_{\ell}}
\newcommand{\gG}{\mathcal G}
\newcommand{\hH}{\mathcal H}
\newcommand{\nN}{\textrm{\sc n}}
\newcommand{\lN}{\textrm{\sc ln}}
\newcommand{\pP}{\mathscr P}
\newcommand{\qQ}{\mathscr Q}
\newcommand{\xX}{\mathcal X}

\newcommand{\ddD}{\mathscr D}


%\newcommand{\R}{{\texttt R}}
\newcommand{\risk}{\mathcal R}
\newcommand{\Remp}{R_{{\rm emp}}}

\newcommand*\diff{\mathop{}\!\mathrm{d}}
\newcommand{\ess}{ \textrm{{\sc ess}} }
\newcommand{\tss}{ \textrm{{\sc tss}} }
\newcommand{\rss}{ \textrm{{\sc rss}} }
\newcommand{\rssr}{ \textrm{{\sc rssr}} }
\newcommand{\ussr}{ \textrm{{\sc ussr}} }
\newcommand{\zdata}{\mathbf{z}_{\mathcal D}}
\newcommand{\Pdata}{P_{\mathcal D}}
\newcommand{\Pdatatheta}{P^{\mathcal D}_{\theta}}
\newcommand{\Zdata}{Z_{\mathcal D}}


\newcommand{\e}[1]{\mathbbm{E}[{#1}]}
\newcommand{\p}[1]{\mathbbm{P}({#1})}
% definition
\BeforeBeginEnvironment{definition}{
  \setbeamerfont{block title}{series=\bfseries}
  \setbeamercolor{block title}{fg=MidnightBlue,bg=white}
  \setbeamercolor{block body}{fg=black, bg=gray!10}
}
\newtheorem*{definition*}{Definition}
\BeforeBeginEnvironment{definition*}{
  \setbeamerfont{block title}{series=\bfseries}
  \setbeamercolor{block title}{fg=MidnightBlue,bg=white}
  \setbeamercolor{block body}{fg=black, bg=gray!10}
}

% theorem
\BeforeBeginEnvironment{theorem}{
  \setbeamerfont{block body}{shape=\itshape}
  \setbeamerfont{block title}{series=\bfseries}
  \setbeamercolor{block title}{fg=MidnightBlue,bg=white}
  \setbeamercolor{block body}{fg=black, bg=gray!10}
}
\newtheorem*{theorem*}{Theorem}
\BeforeBeginEnvironment{theorem*}{
  \setbeamerfont{block body }{shape=\itshape}
  \setbeamerfont{block title}{series=\bfseries}
  \setbeamercolor{block title}{fg=MidnightBlue,bg=white}
  \setbeamercolor{block body}{fg=black, bg=gray!10}
}

% definition_fr
\theoremstyle{definition}
\newtheorem{definition_fr}{Définition}%[section]
\BeforeBeginEnvironment{definition_fr}{
  \setbeamerfont{block title}{series=\bfseries}
  \setbeamercolor{block title}{fg=MidnightBlue,bg=white}
  \setbeamercolor{block body}{fg=black, bg=gray!10}
}
\newtheorem*{definition_fr*}{Définition}
\BeforeBeginEnvironment{definition_fr*}{
  \setbeamerfont{block title}{series=\bfseries}
  \setbeamercolor{block title}{fg=MidnightBlue,bg=white}
  \setbeamercolor{block body}{fg=black, bg=gray!10}
}
% theorem_fr
\newtheorem{theorem_fr}{Théorème}%[section]
\BeforeBeginEnvironment{theorem_fr}{
  \setbeamerfont{block body}{shape=\itshape}
  \setbeamerfont{block title}{series=\bfseries, shape = \upshape}
  \setbeamercolor{block title}{fg=MidnightBlue,bg=white}
  \setbeamercolor{block body}{fg=black, bg=gray!10}
}
\newtheorem*{theorem_fr*}{Théorème}
\BeforeBeginEnvironment{theorem_fr*}{
  \setbeamerfont{block body}{shape=\itshape}
  \setbeamerfont{block title}{series=\bfseries, shape = \upshape}
  \setbeamercolor{block title}{fg=MidnightBlue,bg=white}
  \setbeamercolor{block body}{fg=black, bg=gray!10}
}

% remark_fr
\theoremstyle{remark}
\newtheorem{remark_fr}{Remarque}%[section]
\BeforeBeginEnvironment{remark_fr}{
  \setbeamerfont{block title}{series=\bfseries, shape=\itshape}
  \setbeamercolor{block title}{fg=MidnightBlue,bg=white}
  \setbeamercolor{block body}{fg=black, bg=gray!10}
}
\newtheorem*{remark_fr*}{Remarque}
\BeforeBeginEnvironment{remark_fr*}{
  \setbeamerfont{block title}{series=\bfseries, shape=\itshape}
  \setbeamercolor{block title}{fg=MidnightBlue,bg=white}
  \setbeamercolor{block body}{fg=black, bg=gray!10}
}








\usepackage[svgnames]{xcolor}
\usepackage{tikz}
\usetikzlibrary{shapes.geometric, arrows}
\usepackage{enumerate}   
\usepackage{multirow}
\usepackage{txfonts}
\usepackage{mathrsfs}
\usepackage{pgfplots}
\pgfplotsset{compat = newest}
\usetikzlibrary{positioning, arrows.meta}
\usepgfplotslibrary{fillbetween}
\newcommand{\A}{(0,0) ++(135:2) circle (2)}
\newcommand{\B}{(0,0) ++(45:2) circle (2)}
\DeclareMathOperator{\C}{C}
\DeclareMathOperator{\util}{u}
%\setbeamersize{text margin left=1.5em,text margin right=1.5em} 
%\setbeamersize{text margin left=1.2cm,text margin right=1.2cm} 
\setbeamersize{text margin left=1.5em,text margin right=1.5em} 
%\usepackage{xr}
%\externaldocument{Econometrie1_UGA_P2e}
  \usepackage{eso-pic}
%\newcommand\AtPagemyUpperLeft[1]{\AtPageLowerLeft{%
%\put(\LenToUnit{0.9\paperwidth},\LenToUnit{0.85\paperheight}){#1}}}
%\AddToShipoutPictureFG{
 % \AtPagemyUpperLeft{{\includegraphics[width=1.1cm,keepaspectratio]{logoUGA2020.pdf}}}
%}%

%\setbeamercolor{title}{fg=black}
%\setbeamercolor{frametitle}{fg=black}
%\setbeamercolor{section in head/foot}{fg=black}
%\setbeamercolor{author in head/foot}{bg=Brown}
%\setbeamercolor{date in head/foot}{fg=Brown}
\AtBeginSection[]
  {
    \ifnum \value{framenumber}>1
      \begin{frame}<beamer>
      \frametitle{Plan}
      \tableofcontents[currentsection]
      \end{frame}
    \else
    \fi
  }
\setbeamertemplate{section page}
{
    \begin{centering}
    \begin{beamercolorbox}[sep=11pt,center]{part title}
    \usebeamerfont{section title}\thesection.~\insertsection\par
    \end{beamercolorbox}
    \end{centering}
}

%\titlegraphic{\includegraphics[width=1cm]{logoUGA2020.pdf}}
\title[]{ \textbf{Inférence statistique}}
\subtitle{Cours 4: Méthode de Moments}
\date{\today}
\author{Michal W. Urdanivia\inst{*}}
\institute{\inst{*}UGA, Facult\'e d'\'Economie, GAEL, \\e-mail: \href{mailto:michal.wong-urdanivia@univ-grenoble-alpes.fr}{michal.wong-urdanivia@univ-grenoble-alpes.fr}}

%\titlegraphic{\includegraphics[width=1cm]{logoUGA2020.pdf}
%}
\begin{document}
%%% TIKZ STUFF
\usetikzlibrary{positioning}
\usetikzlibrary{snakes}
\usetikzlibrary{calc}
\usetikzlibrary{arrows}
\usetikzlibrary{decorations.markings}
\usetikzlibrary{shapes.misc}
\usetikzlibrary{matrix,shapes,arrows,fit,tikzmark}
\usetikzlibrary{shapes}
\tikzset{   
        every picture/.style={remember picture,baseline},
        every node/.style={anchor=base,align=center,outer sep=1.5pt},
        every path/.style={thick},
        }
\newcommand\marktopleft[1]{%
    \tikz[overlay,remember picture] 
        \node (marker-#1-a) at (-.3em,.3em) {};%
}
\newcommand\markbottomright[2]{%
    \tikz[overlay,remember picture] 
        \node (marker-#1-b) at (0em,0em) {};%
}
\tikzstyle{every picture}+=[remember picture] 
\tikzstyle{mybox} =[draw=black, very thick, rectangle, inner sep=10pt, inner ysep=20pt]
\tikzstyle{fancytitle} =[draw=black,fill=red, text=white]
\tikzstyle{observed}=[draw,circle,fill=gray!50]



\begin{frame}
\titlepage
\end{frame}
\begin{frame}{Plan}
 \tableofcontents
    \end{frame}
%\begin{frame}
%\frametitle{Contenu}
%\tableofcontents[pausesections, pausesubsections]
%\end{frame}

%\section{Qu'est-ce que l’économétrie ? A quoi (à qui) ça sert ?}
%\frame{\sectionpage}
%\begin{frame}
%  \tableofcontents  
%\end{frame}

\section{Introduction}
\frame{\sectionpage}

\begin{frame}[allowframebreaks]{Théorème d'approximation de  Weierstrass}
    \begin{itemize}
        \item \textbf{\underline{Théorème}}
        \begin{enumerate}[-]
            \item Soit $f$ une fonction continue sur l'intervalle $[a, b]$, alors, pour tout $\epsilon>0$, il existe 
            $a_0, a_1, \ldots, a_d \in \R$ tels que, \begin{align*}
                \max_{x\in[a, b]}\abs{f(x) - \sum_{k=1}^d a_kx^k}&<\epsilon.
            \end{align*}
            \item Autrement dit: \textbf{les fonctions continues peuvent être approchées arbitrairement par des polynômes}.
        \end{enumerate}
    \end{itemize}
\end{frame}

\begin{frame}
    [allowframebreaks]{Application statistique}
    \begin{itemize}
        \item Soit un échantillon de v.a. i.i.d., $X_1, X_2, \ldots, X_n$ associé à un modèle statistique qu'on suppose identifié,
        $\left(\mathcal{E}, \mathcal{F}, (\Prob_\theta)_{\theta\in \Theta}\right)$.
        \item On note $\theta^*$ le vrai paramètre.
        \item Supposons que pour tout $\theta$, la loi $\Prob_\theta$ a la fonction  de densité $f_\theta$.
        \item Si nous trouvons un $\theta$ tel que,
        \begin{align*}
            \int h(x)f_{\theta^*}(x)\mathrm{d}x &=\int h(x)f_{\theta}(x)\mathrm{d}x
        \end{align*}
        pour toutes les fonctions (continues et bornées) $h$, alors $\theta = \theta^*$.
        \item En remplaçant les espérances par des moyennes: il s'agit de trouver un estimateur $\hat{\theta}$ tel que, 
        \begin{align*}
            \frac{1}{n}\sumin h(X_i) &=\int h(x)f_{\hat{\theta}}(x)\mathrm{d}x
        \end{align*}
        pout toutes les fonctions continues et bornées $h$.
        \item \textbf{Problème:} il y a une \textbf{infinité} de fonctions de la sort(infaisable).
        
        \framebreak

        \item Par application du TAW, il suffit de considérer des polynômes: 
        \begin{align*}
            \frac{1}{n}\sumin\sum_{k=0}^d a_k X_i^k &= \int \sum_{k=0}^d a_kx^kf_{\hat{\theta}}(x)\mathrm{d}x, \ \forall a_0, a_1, \ldots, a_d\in \R.
        \end{align*}
        Soit encore une infinité d'équations: 
        \item Pour sa part, il suffit de considérer,
        \begin{align*}
            \frac{1}{n}\sumin X_i^k = \int x^k f_{\hat{\theta}(x)}\mathrm{d}x, \forall k=1, 2, \ldots, d,
        \end{align*}
        (seulement $d+1$ équations).
        \item La quantité $m_k(\theta):= \int x^kf_\theta(x)\mathrm{d}x$ est le moment d'ordre $k$(ou $k$-ième moment) de $\Prob_\theta$.
        \item On peut aussi l'écrire $m_k(\theta)=\Exp_\theta(X^k)$. 
    \end{itemize}
\end{frame}
\begin{frame}
    [allowframebreaks]{Quadrature de Gauss}
    \begin{itemize}
        \item L'approximation de Weierstrass présente un certain nombre de limitations: \begin{enumerate}[i)]
            \item ne s'applique qu'à des fonctions continues(ceci peut être réglé),
            \item ne s'applique que sur des intervalles $[a,b]$,
            \item ne nous dit pas ce que $d$ doit être(i.e., $\#$ de moments).
        \end{enumerate}
        \item Qu'en est-il pour $\mathcal{E}$ discret(cas où à $\Prob_\theta$ on 
        associe une fonction de masse et non de densité)?
        \framebreak
        \item Supposons que $\mathcal{E} = \{x_1, x_2, \ldots, x_r\}$ est fini avec $r$ valeurs possibles.
        \item La fonction de masse est alors caractérisée par $r-1$ paramètres,
        \[
        \prob(x_1), \prob(x_2), \ldots, \prob(x_{r-1}),
        \]
        car le dernier est donné par les premiers $r-1$ paramètres,
        \[
        \prob_r = 1-\sum_{j=1}^{r-1}\prob(x_j).
        \]
        \item Avec un peu de chance, il ne faudra pas plus que $d=r-1$
         paramètres pour obtenir la fonction de masse $\prob(\cdot)$.

         \framebreak
         \item Notons que pour tout $k=1, 2, \ldots, r_1$,
         \begin{align*}
             m_k &:=\Exp(X^k) = \sum_{j=1}^r\prob(x_j)x_j^k,
         \end{align*}
         et,
         \begin{align*}
             \sum_{j=1}^r\prob(x_j)&=1
         \end{align*}
         \item Ceci est un système d'équations linéaires avec les inconnues, $\prob(x_1), \prob(x_2), \ldots, \prob(x_r)$.
         \item Il est peut être écrit sous une forme compacte(matricielle): 
         \begin{align*}
            \begin{pmatrix}
                1&1&\ldots&1\\
                x_1^1&x_2^1&\ldots&x_r^1\\
                x_1^2&x_2^2&\ldots&x_r^2\\
                \vdots&&\ddots&\vdots\\
                x_1^{r-1}&x_2^{r-1}&\ldots&x_r^{r-1}
             \end{pmatrix} \cdot \begin{pmatrix}
                \prob(x_1)\\
                \prob(x_2)\\
                \vdots\\
                \prob(x_{r-1})\\
                \prob(x_r)
             \end{pmatrix}
             &=\begin{pmatrix}
                1\\
                 m_1\\
                 m_2\\
                 \vdots\\
                 m_{r-1}
             \end{pmatrix}
         \end{align*}
         \framebreak
         \item Il faut vérifier que l'inversibilité: \textbf{déterminant de Vandermonde}
         \begin{align*}
            \det\begin{pmatrix}
                1&1&\ldots&1\\
                x_1^1&x_2^1&\ldots&x_r^1\\
                x_1^2&x_2^2&\ldots&x_r^2\\
                \vdots&&\ddots&\vdots\\
                x_1^{r-1}&x_2^{r-1}&\ldots&x_r^{r-1}
             \end{pmatrix} 
             &= \prod_{1<j<k<r}(x_j-x_k) \neq 0
         \end{align*}
         \framebreak
         \item Donc, si nous avons les moments $m_1, m_2, \ldots, m_{r-1}$, il y a une seule fonction de masse avec ces moments.
         Et elle est donnée par,
         \begin{align*}
             \begin{pmatrix}
                \prob(x_1)\\
                \prob(x_2)\\
                \vdots\\
                \prob(x_{r-1})
             \end{pmatrix}
             &=
             {
             \begin{pmatrix}
                1&1&\ldots&1\\
                x_1^1&x_2^1&\ldots&x_r^1\\
                x_1^2&x_2^2&\ldots&x_r^2\\
                \vdots&&\ddots&\vdots\\
                x_1^{r-1}&x_2^{r-1}&\ldots&x_r^{r-1}
             \end{pmatrix} }^{-1}
             \cdot
             \begin{pmatrix}
                1\\
                 m_1\\
                 m_2\\
                 \vdots\\
                 m_{r-1}
             \end{pmatrix}
         \end{align*}
    \end{itemize}
\end{frame}
\begin{frame}
    [allowframebreaks]{Conclusion à partir du TAW et de la quadrature de Gauss}
    \begin{itemize}
        \item Les moments d'une v.a., fournissent une information importante sur la fonction de densité ou de masse.
        \item En estimant avec précision ces moments on devrait être capables de récupérer/retrouver la loi génératrice des données.
        \item Dans un cadre paramétrique où la loi des observations $\Prob_\theta$ est connue aux paramètres $\theta$ près, il est fréquent de n'avoir 
        besoin que d'un petit nombre de moments pour obtenir(estimer) $\theta$, ceci variant d'un cas à l'autre.
        \item Règle basique: quand $\theta\in \Theta\subseteq\R^d$, on a besoin de $d$ moments.
    \end{itemize}
\end{frame}

\section{Méthode des Moments}
\frame{\sectionpage}

\begin{frame}[allowframebreaks]{Principe d'estimation de la MM}
\begin{itemize}
    \item Soit $X_1, X_2, \ldots, X_n$ un échantillon i.i.d. pour le modèle statistique 
    $\left(\mathcal{E}, \mathcal{F}, (\Prob_\theta)_{\theta\in\Theta}\right)$. 
    \item On suppose $\Theta\subseteq \R^d$, pour un $d\geq 1$.
    \begin{enumerate}[-]
        \item \textbf{Moments théorique(ou de Population):} 
        \begin{align*}
            m_k(\theta) &:=\Exp_\theta(X^k), \quad 1\leq k\leq d.
        \end{align*}
        \item \textbf{Moments empiriques:}
        \begin{align*}
            \hat{m}_k :=\frac{1}{n}\sumin X_i^k, \quad 1\leq k\leq d.
        \end{align*}
    \end{enumerate}
    \framebreak
    \item Soit,\begin{align*}
        \psi : \Theta \subseteq\R^d &\rightarrow \R^d\\
        \theta&\mapsto (m_1(\theta), m_2(\theta), \ldots, m_d(\theta)).
    \end{align*}
    \item Supposons $\psi$ bijective, alors,\begin{align*}
        \theta &= \psi^{-1}(m_1(\theta), m_2(\theta), \ldots, m_d(\theta)).
    \end{align*}
    \item \textbf{\underline{Définition:}} un estimateur des moments de $\theta$ est défini par, \begin{align*}
        \hat{\theta}^{MM}_n &=\psi^{-1}(\hat{m}_1, \hat{m}_2, \ldots, \hat{m}_d),
    \end{align*}
    dès lors qu'il existe.
\end{itemize}
\end{frame}
\begin{frame}
    [allowframebreaks]{Analyse de $\hat{\theta}_n^{MM}$}\begin{itemize}
        \item Notons: \begin{align*}
            M(\theta) &:= \left(m_1(\theta), m_2(\theta), \ldots, m_d(\theta)\right),\\
            \hat{M} &=\left(\hat{m}_1, \hat{m}_2, \ldots, \hat{m}_d \right),\\
            \Vr(\theta) &=\underbrace{\Var_{\theta}\left(X, X^2, \ldots, X^d\right)}_{\text{Matrice des variances-covariances de $(X, X^2, \ldots, X^d)$, quand $X\sim\Prob_\theta$.}}
        \end{align*}.
        \item Supposons $\psi^{-1}$ continûment dérivable en $M(\theta)$. Notons $\nabla\psi^{-1}_{M(\theta)}$ la matrice $d\times d$ du gradient en ce point.
        \framebreak 
        \item \textbf{LGN}: $\hat{\theta}_n^{MM}$ est faiblement/fortement convergent.
        \item \textbf{TCL}: \begin{align*}
            \sqrt{n}\left(\hat{M}- M(\theta)\right)&\limd \mathcal{N}(0, \Vr(\theta)) \quad \text{par rapport à $\Prob_\theta$.}
        \end{align*}
        et en utilisant la méthode du delta(voir diapos suivantes),
        \framebreak
        \item \textbf{\underline{Théorème}}:\begin{align*}
            \sqrt{n}\left(\hat{\theta}^{MM}_n - \theta\right) &\limd\mathcal{N}(0, \Gamma(\theta)) \quad \text{par rapport à $\Prob_\theta$},
        \end{align*}
        où \begin{align*}
            \Gamma(\theta)) = \left(\nabla\psi^{-1}_{M(\theta)}\right)^\top \Vr(\theta)\left(\nabla\psi^{-1}_{M(\theta)}\right).
        \end{align*}
    \end{itemize}
\end{frame}
\begin{frame}[allowframebreaks]{Méthode du Delta dans le cas multivarié}
\begin{itemize}
    \item Soit $(T_n)_{n\geq 1}$ une suite de vecteurs aléatoires dans $\R^p$(pour $p\geq 1$) telle que,\begin{align*}
        \sqrt{n}\left(T_n - \theta\right)&\limd\mathcal{N}(0, \Vr),
    \end{align*}
    pour un $\theta \in \R^p$ et une matrice symétrique et semi-définie positive $\Vr\in \R^p\times p$.
    \item Soit $g: \R^p\rightarrow\R^k$(pour $k\geq 1$) une fonction continûment dérivable en $\theta$. 
    \item Alors,\begin{align*}
        \sqrt{n}\left(g(T_n) - g(\theta)\right) &\limd 
        \mathcal{N}\left(0, \nabla g(\theta)^\top\Vr \nabla g(\theta)\right),
    \end{align*}
    où,\begin{align*}
        \nabla g(\theta) := \left(\frac{\partial g_j}{\partial \theta_i}\right)_{1\leq i\leq d,  1\leq j\leq k}\in\R^{k\times d}.
    \end{align*}
\end{itemize}    
\end{frame}
\begin{frame}
    [allowframebreaks]{MV vs MM}\begin{itemize}
        \item \textbf{Risque quadratique}: en général l'estimateur du MV est plus précis.
        \item \textbf{Problème de calcul}: l'estimateur du MV ne peut pas être obtenu en forme analytique: 
        \begin{enumerate}[-]
            \item Quand la vraisemblance est concave, on peut utiliser des algorithmes d'optimisation numérique(
                méthodes du point intérieur, gradient descendant, etc).
            \item Si la vraisemblance n'est pas concave: démarche heuristique, maxima locaux.
        \end{enumerate}
    \end{itemize}
\end{frame}

\end{document}