%\documentclass[ignorenonframetext, compress, 9pt, xcolor=svgnames]{beamer} 
\documentclass[notes, ignorenonframetext, compress, 11pt, xcolor=svgnames, aspectratio=169]{beamer} 
\usepackage{pgfpages}
\usepackage{pdfpages}
% These slides also contain speaker notes. You can print just the slides,
% just the notes, or both, depending on the setting below. Comment out the want
% you want.
\setbeameroption{hide notes} % Only slide
%\setbeameroption{show only notes} % Only notes
%\setbeameroption{show notes on second screen=right} % Both
\usepackage{amsmath}
\usepackage{amsfonts}
\usepackage{amssymb}
\setbeamercolor{frametitle}{fg=MidnightBlue}

\setbeamercolor{sectionpage title}{bg=MidnightBlue}
\setbeamertemplate{frametitle}[default][center]
%\setbeamertemplate{frametitle}{\color{MidnightBlue}\centering\bfseries\insertframetitle\par\vskip-6pt}
\setbeamerfont{frametitle}{series=\bfseries}
\setbeamerfont{title}{series=\bfseries}
\setbeamerfont{sectionpage}{series=\bfseries}
%\setbeamercolor{section in head/foot}{bg=MidnightBlueBlue}
%\setbeamercolor{author in head/foot}{bg=DarkBlue}
\setbeamercolor{author in head/foot}{fg=MidnightBlue}
%\setbeamercolor{title in head/foot}{bg=White}
\setbeamercolor{title in head/foot}{fg=MidnightBlue}
\setbeamercolor{title}{fg=MidnightBlue}
%\setbeamercolor{date in head/foot}{fg=Brown}
%\setbeamercolor{alerted text}{fg=DarkBlue}
%\usecolortheme[named=DarkBlue]{structure} 
%\usepackage{bbm}
%\usepackage{bbold}
\usepackage{eurosym}
\usepackage{graphicx}
%\usepackage{epstopdf}
\usepackage{hyperref}
\hypersetup{
  colorlinks   = true, %Colours links instead of ugly boxes
  urlcolor     = gray, %Colour for external hyperlinks
  linkcolor    = MidnightBlue, %Colour of internal links
  citecolor   = DarkRed %Colour of citations
}
\usepackage{multirow}
\usepackage{xspace}
\usepackage{listings}
\usepackage{natbib}
%\usepackage[sort&compress,comma,super]{natbib}
\def\newblock{} % To avoid a compilation error about a function \newblock undefined
\usepackage{bibentry}
\usepackage{booktabs}
\usepackage{dcolumn}
\usepackage[greek,frenchb]{babel}
\usepackage[babel=true,kerning=true]{microtype}
\usepackage[utf8]{inputenc}
\usepackage[T1]{fontenc}
\usepackage{natbib}
\renewcommand{\cite}{\citet}
\usepackage{longtable}
\usepackage{eso-pic}

\usepackage{xcolor}
 \colorlet{linkequation}{DarkRed} 
 \newcommand*{\SavedEqref}{}
 \let\SavedEqref\eqref 
\renewcommand*{\eqref}[1]{%
\begingroup \hypersetup{
      linkcolor=linkequation,
linkbordercolor=linkequation, }%
\SavedEqref{#1}%
 \endgroup
}

\newcommand*{\refeq}[1]{%
 \begingroup
\hypersetup{ 
linkcolor=linkequation, 
linkbordercolor=linkequation,
}%
\ref{#1}%
 \endgroup
}

\setbeamertemplate{caption}[numbered]
\setbeamertemplate{theorem}[ams style]
\setbeamertemplate{theorems}[numbered]
%\usefonttheme{serif}
%\usecolortheme{beaver}
%\usetheme{Hannover}
%\usetheme{CambridgeUS}
%\usetheme{Madrid}
%\usecolortheme{whale}
%\usetheme{Warsaw}
%\usetheme{Luebeck}
%\usetheme{Montpellier}
%\usetheme{Berlin}
%\setbeamercolor{titlelike}{parent=structure}
%\setbeamertemplate{headline}[default]
%\setbeamertemplate{footline}[default]
%\setbeamertemplate{footline}[Malmoe]
%\setbeamercovered{transparent}
%\setbeamercovered{invisible}
%\usecolortheme{crane}
%\usecolortheme{dolphin}
%\usepackage{pxfonts}
%\usepackage{isomath}
%\usepackage{mathpazo}
%\usepackage{arev} %     (Arev/Vera Sans)
%\usepackage{eulervm} %_   (Euler Math)
%\usepackage{fixmath} %  (Computer Modern)
%\usepackage{hvmath} %_   (HV-Math/Helvetica)
%\usepackage{tmmath} %_   (TM-Math/Times)
%\usepackage{tgheros}
%\usepackage{cmbright}
%\usepackage{ccfonts} \usepackage[T1]{fontenc}
%\usepackage[garamond]{mathdesign}

%\usepackage{color}
%\usepackage{ulem}

%\usepackage[math]{kurier}
%\usepackage[no-math]{fontspec}
%\setmainfont{Fontin Sans}
%\setsansfont{Fontin Sans}
%\setbeamerfont{frametitle}{size=\LARGE,series=\bfseries}
%%%add 19022021
\usepackage{enumerate}    
\usepackage{dcolumn}
\usepackage{verbatim}
\newcolumntype{d}[0]{D{.}{.}{5}}
%\setbeamertemplate{note page}{\pagecolor{yellow!5}\insertnote}
%\usetikzlibrary{positioning}
%\usetikzlibrary{snakes}
%\usetikzlibrary{calc}
%\usetikzlibrary{arrows}
%\usetikzlibrary{decorations.markings}
%\usetikzlibrary{shapes.misc}
%\usetikzlibrary{matrix,shapes,arrows,fit,tikzmark}
%%%
% suppress navigation bar
\beamertemplatenavigationsymbolsempty
%\usetheme{bunsenMod}
%\setbeamercovered{transparent}
%\setbeamertemplate{items}[circle]
%\usecolortheme[named=CadetBlue]{structure}
%\usecolortheme[RGB={225,64,5}]{structure}
%\definecolor{burntRed}{RGB}{225,64,5}
%\setbeamercolor{alerted text}{fg=burntRed} 
%\usecolortheme[RGB={0,40,110}]{structure}
%\hypersetup{linkcolor=burntRed}
%\hypersetup{urlcolor=burntRed}
%\hypersetup{filecolor=burntRed}
%\hypersetup{citecolor=burntRed}

%\usetheme{bunsenMod}
%\setbeamercovered{transparent}
%\setbeamertemplate{items}[circle]
%\usecolortheme[named=CadetBlue]{structure}
%\usecolortheme[RGB={225,64,5}]{structure}
%\definecolor{burntRed}{RGB}{225,64,5}
%\setbeamercolor{alerted text}{fg=burntRed} 
%\usecolortheme[RGB={0,40,110}]{structure}
%\hypersetup{linkcolor=burntRed}
%\hypersetup{urlcolor=burntRed}
%\hypersetup{filecolor=burntRed}
%\hypersetup{citecolor=burntRed}

%\AtBeginSection[] % Do nothing for \section*
%{ \frame{\sectionpage} }
%\setbeamertemplate{frametitle continuation}{}
\newtheorem{lemme}{Lemme}[section]
%\newtheorem{remarque}{Remarque}
\newcommand{\argmax}{\operatornamewithlimits{arg\,max}}
\newcommand{\argmin}{\operatornamewithlimits{arg\,min}}
\def\inprobLOW{\rightarrow_p}
\def\inprobHIGH{\,{\buildrel p \over \rightarrow}\,} 
\def\inprob{\,{\inprobHIGH}\,} 
\def\indist{\,{\buildrel d \over \rightarrow}\,} 
\def\sima{\,{\buildrel a \over \sim}\,} 
\def\F{\mathbb{F}}
\def\R{\mathbb{R}}
\def\N{\mathbb{N}}
\newcommand{\gmatrix}[1]{\begin{pmatrix} {#1}_{11} & \cdots &
    {#1}_{1n} \\ \vdots & \ddots & \vdots \\ {#1}_{m1} & \cdots &
    {#1}_{mn} \end{pmatrix}}
\newcommand{\iprod}[2]{\left\langle {#1} , {#2} \right\rangle}
\newcommand{\norm}[1]{\left\Vert {#1} \right\Vert}
\newcommand{\abs}[1]{\left\vert {#1} \right\vert}
\renewcommand{\det}{\mathrm{det}}
\newcommand{\rank}{\mathrm{rank}}
\newcommand{\spn}{\mathrm{span}}
\newcommand{\row}{\mathrm{Row}}
\newcommand{\col}{\mathrm{Col}}
\renewcommand{\dim}{\mathrm{dim}}
\newcommand{\prefeq}{\succeq}
\newcommand{\pref}{\succ}
\newcommand{\seq}[1]{\{{#1}_n \}_{n=1}^\infty }
\renewcommand{\to}{{\rightarrow}}
\renewcommand{\L}{{\mathcal{L}}}
\newcommand{\Er}{\mathrm{E}}
\renewcommand{\Pr}{\mathrm{P}}
%\newcommand{\Var}{\mathrm{Var}}
%\newcommand{\Cov}{\mathrm{Cov}}
%\newcommand{\corr}{\mathrm{Corr}}
%\newcommand{\Var}{\mathrm{Var}}
\newcommand{\bias}{\mathrm{Bias}}
\newcommand{\mse}{\mathrm{MSE}}
\providecommand{\Pred}{\mathcal{P}}
\providecommand{\plim}{\operatornamewithlimits{plim}}
\providecommand{\avg}{\frac{1}{n} \underset{i=1}{\overset{n}{\sum}}}
\providecommand{\sumin}{{\sum_{i=1}^n}}
\providecommand{\limp}{\overset{p}{\rightarrow}}
%\providecommand{\limp}{\underset{n \rightarrow \infty}{\overset{p}{\longrightarrow}}}
%\providecommand{\limp}{\underset{n \rightarrow \infty}{\overset{p}{\longrightarrow}}}
%\providecommand{\limp}{\overset{p}{\longrightarrow}}
%\providecommand{\limd}{\underset{n \rightarrow \infty}{\overset{d}{\longrightarrow}}}
\providecommand{\limd}{\overset{d}{\rightarrow}}
\providecommand{\limps}{\overset{p.s.}{\rightarrow}}
\providecommand{\limlp}{\overset{L^p}{\rightarrow}}
\def\independenT#1#2{\mathrel{\setbox0\hbox{$#1#2$}%
    \copy0\kern-\wd0\mkern4mu\box0}} 
\newcommand\indep{\protect\mathpalette{\protect\independenT}{\perp}}


\lstset{language=R}
\lstset{keywordstyle=\color[rgb]{0,0,1},                                        % keywords
        commentstyle=\color[rgb]{0.133,0.545,0.133},    % comments
        stringstyle=\color[rgb]{0.627,0.126,0.941}      % strings
}       
\lstset{
  showstringspaces=false,       % not emphasize spaces in strings 
  columns=fixed,
  numbersep=3mm, numbers=left, numberstyle=\tiny,       % number style
  frame=none,
  framexleftmargin=5mm, xleftmargin=5mm         % tweak margins
}
\makeatletter
%\setbeamertemplate{frametitle continuation}{\gdef\beamer@frametitle{}}
\setbeamertemplate{frametitle continuation}{\frametitle{}}
%\setbeamertemplate{frametitle continuation}{\insertcontinuationcount}
\makeatother

\theoremstyle{remark}
\newtheorem{interpretation}{Interprétation}
\newtheorem*{interpretation*}{Interprétation}

\theoremstyle{remark}
\newtheorem{remarque}{Remarque}%[section]
\newtheorem*{remarque*}{Remarque}
\usepackage[framemethod=TikZ]{mdframed} 
\usepackage{showexpl}
%\newtheorem{step}{Step}[section]
%\newtheorem{rem}{Comment}[section]
%\newtheorem{ex}{Example}[section]
%\newtheorem{hist}{History}[section]
%\newtheorem*{ex*}{Example}
\theoremstyle{plain}
\newtheorem{propriete}{Propri\'et\'e}
\renewcommand{\thepropriete}{P\arabic{propriete}}
%\theoremstyle{definition}
%\newtheorem{definition}{Définition}%[section]
\theoremstyle{remark}
\newtheorem{exemple}{Exemple}
\newtheorem*{exemple*}{Exemple}

\newtheorem{theoreme}{Théorème}
\newtheorem{proposition}{Proposition}
%\newtheorem{propriete}{Propri\'et\'e}
\newtheorem{corollaire}{Corollaire}
%\newtheorem{exemple}{Exemple}
\newtheorem{assumption}{Assumption}
\renewcommand{\theassumption}{A\arabic{assumption}}
\newtheorem{hypothese}{Hypothèse}
\renewcommand{\thehypothese}{H\arabic{hypothese}}
\theoremstyle{definition}

%\newtheorem{definitionx}{D\'efinition}%[section]
%\newenvironment{definition}
 %{\pushQED{\qed}\renewcommand{\qedsymbol}{$\triangle$}\definitionx}
 %{\popQED\enddefinitionx}

\newtheorem{condition}{Condition}
\renewcommand{\thecondition}{C\arabic{condition}}
%\newcommand{\Var}{\mathbb{V}}
%\newcommand{\Var}{\mathbf{Var}}
%\newcommand{\Exp}{\mathbf{E}}
%\providecommand{\Vr}{\mathrm{Var}}
%\renewcommand{\Er}{\mathbb{E}}
%\newcommand{\LP}{\mathcal{LP}}
%\providecommand{\Id}{\mathbf{I}}
%\providecommand{\Rang}{\mathrm{Rang}}
%\providecommand{\Trace}{\mathrm{Trace}}
%\newcommand{\Cov}{\mathbf{Cov}}
%\newcommand{\Cov}{\mathbb{C}\mathrm{ov}}
\providecommand{\Id}{\mathbf{I}}
\providecommand{\Ind}{\mathbf{1}}
\providecommand{\uvec}{\mathbf{1}}
\providecommand{\vecOnes}{\mathbf{1}}
\DeclareMathOperator{\indfun}{\mathbf{1}}
\DeclareMathOperator{\Exp}{E}
\DeclareMathOperator{\Expn}{\mathbb{E}_n}
\DeclareMathOperator{\Var}{Var}
\DeclareMathOperator{\Vr}{V}
\DeclareMathOperator{\Cov}{Cov}
\DeclareMathOperator{\corr}{corr}
\DeclareMathOperator{\perps}{\perp_s}
%\DeclareMathOperator{\Prob}{Pr}
\DeclareMathOperator{\Prob}{P}
\DeclareMathOperator{\prob}{p}
\DeclareMathOperator{\loss}{L}
\providecommand{\Corr}{\mathrm{Corr}}
\providecommand{\Diag}{\mathrm{Diag}}
\providecommand{\reg}{\mathrm{r}}
\providecommand{\Likelihood}{\mathrm{L}}
\renewcommand{\Pr}{{\mathbb{P}}}
\providecommand{\set}[1]{\left\{#1\right\}}
\providecommand{\uvec}{\mathbf{1}}
\providecommand{\Rang}{\mathrm{Rang}}
\providecommand{\Trace}{\mathrm{Trace}}
\providecommand{\Tr}{\mathrm{Tr}}
\providecommand{\CI}{\mathrm{CI}}
\providecommand{\asyvar}{\mathrm{AsyVar}}
\DeclareMathOperator{\Supp}{Supp}
\newcommand{\inputslide}[2]{{
    \usebackgroundtemplate{
     \includegraphics[page={#2},width=0.90\textwidth,keepaspectratio=true]
      %\includegraphics[page={#2},width=\paperwidth,keepaspectratio=true]
      {{#1}}}
    \frame[plain]{}
  }}
\newcommand\pperp{\perp\!\!\!\perp}
\newcommand\independent{\protect\mathpalette{\protect\independenT}{\perp}}
\def\independenT#1#2{\mathrel{\rlap{$#1#2$}\mkern2mu{#1#2}}}
\usepackage{bbm}
\providecommand{\Ind}{\mathbf{1}}
\newcommand{\sumjsi}{\underset{i<j}{{\sum}}}
\newcommand{\prodjsi}{\underset{i<j}{{\prod}}}
\newcommand{\sumisj}{\underset{j<i}{{\sum}}}
\newcommand{\prodisj}{\underset{j<i}{{\prod}}}
\newcommand{\sumobs}{\underset{i=1}{\overset{n}{\sum}}}
\newcommand{\sumi}{\underset{i=1}{\overset{n}{\sum}}}
\newcommand{\prodi}{\underset{i=1}{\overset{n}{\prod}}}
\newcommand{\prodobs}{\underset{i=1}{\overset{n}{\prod}}}
\newcommand{\simiid}{{\overset{i.i.d.}{\sim}}}
%\newcommand{\sumobs}{\sum_{i=1}^N}
%\newcommand{\prodobs}{\prod_{i=1}^N}
%\newcommand{\sumjsi}{\sum_{i<j}}
%\newcommand{\prodjsi}{\prod_{i<j}}
%\newcommand{\sumisj}{\sum_{j<i}}
%\newcommand{\prodisj}{\sum_{j<i}}

%\usepackage{appendixnumberbeamer}
\setbeamertemplate{footline}[frame number]
\setbeamertemplate{section in toc}[sections numbered]
\setbeamertemplate{subsection in toc}[subsections numbered]
\setbeamertemplate{subsubsection in toc}[subsubsections numbered]

%\makeatother
%\setbeamertemplate{footline}
%{
%    \leavevmode%
%    \hbox{%
%        \begin{beamercolorbox}[wd=.333333\paperwidth,ht=2.25ex,dp=1ex,center]{author in head/foot}%
%            \usebeamerfont{author in head/foot}\insertshortauthor
%        \end{beamercolorbox}%
%        \begin{beamercolorbox}[wd=.333333\paperwidth,ht=2.25ex,dp=1ex,center]{title in head/foot}%
%            \usebeamerfont{title in head/foot}\insertshorttitle
%        \end{beamercolorbox}%
%        \begin{beamercolorbox}[wd=.333333\paperwidth,ht=2.25ex,dp=1ex,right]{date in head/foot}%
%            \usebeamerfont{date in head/foot}\insertshortdate{}\hspace*{2em}
%            \insertframenumber{} / \inserttotalframenumber\hspace*{2ex} 
%        \end{beamercolorbox}}%
%       \vskip0pt%
 %   }
%   \makeatother
%\setbeamertemplate{navigation symbols}{}
\setbeamertemplate{itemize items}[ball]
%\setbeamertemplate{itemize items}{-}
%\newenvironment{wideitemize}{\itemize\addtolength{\itemsep}{10pt}}{\enditemize}
% \usepackage{eso-pic}
%\newcommand\AtPagemyUpperLeft[1]{\AtPageLowerLeft{%
%\put(\LenToUnit{0.9\paperwidth},\LenToUnit{0.9\paperheight}){#1}}}
%\AddToShipoutPictureFG{
%  \AtPagemyUpperLeft{{\includegraphics[width=1.1cm,keepaspectratio]{../logo-uga.png}}}
%}%
\def\figheight{3in}
\def\figwidth{4in}

%%Commands from Econometric Theory(Slides) by J. Stachurski.

\newcommand{\boldx}{ {\mathbf x} }
\newcommand{\boldu}{ {\mathbf u} }
\newcommand{\boldv}{ {\mathbf v} }
\newcommand{\boldw}{ {\mathbf w} }
\newcommand{\boldy}{ {\mathbf y} }
\newcommand{\boldb}{ {\mathbf b} }
\newcommand{\bolda}{ {\mathbf a} }
\newcommand{\boldc}{ {\mathbf c} }
\newcommand{\boldd}{ {\mathbf d} }
\newcommand{\boldi}{ {\mathbf i} }
\newcommand{\bolde}{ {\mathbf e} }
\newcommand{\boldp}{ {\mathbf p} }
\newcommand{\boldq}{ {\mathbf q} }
\newcommand{\bolds}{ {\mathbf s} }
\newcommand{\boldt}{ {\mathbf t} }
\newcommand{\boldz}{ {\mathbf z} }
\newcommand{\boldr}{ {\mathbf r} }

\newcommand{\boldzero}{ {\mathbf 0} }
\newcommand{\boldone}{ {\mathbf 1} }

\newcommand{\boldalpha}{ {\boldsymbol \alpha} }
\newcommand{\boldbeta}{ {\boldsymbol \beta} }
\newcommand{\boldgamma}{ {\boldsymbol \gamma} }
\newcommand{\boldtheta}{ {\boldsymbol \theta} }
\newcommand{\boldxi}{ {\boldsymbol \xi} }
\newcommand{\boldtau}{ {\boldsymbol \tau} }
\newcommand{\boldepsilon}{ {\boldsymbol \epsilon} }
\newcommand{\boldmu}{ {\boldsymbol \mu} }
\newcommand{\boldSigma}{ {\boldsymbol \Sigma} }
\newcommand{\boldOmega}{ {\boldsymbol \Omega} }
\newcommand{\boldPhi}{ {\boldsymbol \Phi} }
\newcommand{\boldLambda}{ {\boldsymbol \Lambda} }
\newcommand{\boldphi}{ {\boldsymbol \phi} }

\newcommand{\Sigmax}{ {\boldsymbol \Sigma_{\boldx}}}
\newcommand{\Sigmau}{ {\boldsymbol \Sigma_{\boldu}}}
\newcommand{\Sigmaxinv}{ {\boldsymbol \Sigma_{\boldx}^{-1}}}
\newcommand{\Sigmav}{ {\boldsymbol \Sigma_{\boldv \boldv}}}

\newcommand{\hboldx}{ \hat {\mathbf x} }
\newcommand{\hboldy}{ \hat {\mathbf y} }
\newcommand{\hboldb}{ \hat {\mathbf b} }
\newcommand{\hboldu}{ \hat {\mathbf u} }
\newcommand{\hboldtheta}{ \hat {\boldsymbol \theta} }
\newcommand{\hboldtau}{ \hat {\boldsymbol \tau} }
\newcommand{\hboldmu}{ \hat {\boldsymbol \mu} }
\newcommand{\hboldbeta}{ \hat {\boldsymbol \beta} }
\newcommand{\hboldgamma}{ \hat {\boldsymbol \gamma} }
\newcommand{\hboldSigma}{ \hat {\boldsymbol \Sigma} }

\newcommand{\boldA}{\mathbf A}
\newcommand{\boldB}{\mathbf B}
\newcommand{\boldC}{\mathbf C}
\newcommand{\boldD}{\mathbf D}
\newcommand{\boldI}{\mathbf I}
\newcommand{\boldL}{\mathbf L}
\newcommand{\boldM}{\mathbf M}
\newcommand{\boldP}{\mathbf P}
\newcommand{\boldQ}{\mathbf Q}
\newcommand{\boldR}{\mathbf R}
\newcommand{\boldX}{\mathbf X}
\newcommand{\boldU}{\mathbf U}
\newcommand{\boldV}{\mathbf V}
\newcommand{\boldW}{\mathbf W}
\newcommand{\boldY}{\mathbf Y}
\newcommand{\boldZ}{\mathbf Z}

\newcommand{\bSigmaX}{ {\boldsymbol \Sigma_{\hboldbeta}} }
\newcommand{\hbSigmaX}{ \mathbf{\hat \Sigma_{\hboldbeta}} }
\newcommand{\betahat}{\hat{\beta}}
\newcommand{\gammahat}{\hat{\gamma}}
\newcommand{\Uhat}{\hat{U}}
\newcommand{\Vhat}{\hat{V}}
\newcommand{\epsilonhat}{\hat{\epsilon}}
\newcommand{\sigmahat}{\hat{\sigma}}
\newcommand{\Sigmahat}{\hat{\Sigma}}
\newcommand{\Gammahat}{\hat{\Gamma}}

\newcommand{\RR}{\mathbbm R}
\newcommand{\CC}{\mathbbm C}
\newcommand{\NN}{\mathbbm N}
\newcommand{\PP}{\mathbbm P}
\newcommand{\EE}{\mathbbm E \nobreak\hspace{.1em}}
\newcommand{\EEP}{\mathbbm E_P \nobreak\hspace{.1em}}
\newcommand{\ZZ}{\mathbbm Z}
\newcommand{\QQ}{\mathbbm Q}


\newcommand{\XX}{\mathcal X}

\newcommand{\aA}{\mathcal A}
\newcommand{\fF}{\mathscr F}
\newcommand{\bB}{\mathscr B}
\newcommand{\iI}{\mathscr I}
\newcommand{\rR}{\mathscr R}
\newcommand{\dD}{\mathcal D}
\newcommand{\lL}{\mathcal L}
\newcommand{\llL}{\mathcal{H}_{\ell}}
\newcommand{\gG}{\mathcal G}
\newcommand{\hH}{\mathcal H}
\newcommand{\nN}{\textrm{\sc n}}
\newcommand{\lN}{\textrm{\sc ln}}
\newcommand{\pP}{\mathscr P}
\newcommand{\qQ}{\mathscr Q}
\newcommand{\xX}{\mathcal X}

\newcommand{\ddD}{\mathscr D}


%\newcommand{\R}{{\texttt R}}
\newcommand{\risk}{\mathcal R}
\newcommand{\Remp}{R_{{\rm emp}}}

\newcommand*\diff{\mathop{}\!\mathrm{d}}
\newcommand{\ess}{ \textrm{{\sc ess}} }
\newcommand{\tss}{ \textrm{{\sc tss}} }
\newcommand{\rss}{ \textrm{{\sc rss}} }
\newcommand{\rssr}{ \textrm{{\sc rssr}} }
\newcommand{\ussr}{ \textrm{{\sc ussr}} }
\newcommand{\zdata}{\mathbf{z}_{\mathcal D}}
\newcommand{\Pdata}{P_{\mathcal D}}
\newcommand{\Pdatatheta}{P^{\mathcal D}_{\theta}}
\newcommand{\Zdata}{Z_{\mathcal D}}


\newcommand{\e}[1]{\mathbbm{E}[{#1}]}
\newcommand{\p}[1]{\mathbbm{P}({#1})}
% definition
\BeforeBeginEnvironment{definition}{
  \setbeamerfont{block title}{series=\bfseries}
  \setbeamercolor{block title}{fg=MidnightBlue,bg=white}
  \setbeamercolor{block body}{fg=black, bg=gray!10}
}
\newtheorem*{definition*}{Definition}
\BeforeBeginEnvironment{definition*}{
  \setbeamerfont{block title}{series=\bfseries}
  \setbeamercolor{block title}{fg=MidnightBlue,bg=white}
  \setbeamercolor{block body}{fg=black, bg=gray!10}
}

% theorem
\BeforeBeginEnvironment{theorem}{
  \setbeamerfont{block body}{shape=\itshape}
  \setbeamerfont{block title}{series=\bfseries}
  \setbeamercolor{block title}{fg=MidnightBlue,bg=white}
  \setbeamercolor{block body}{fg=black, bg=gray!10}
}
\newtheorem*{theorem*}{Theorem}
\BeforeBeginEnvironment{theorem*}{
  \setbeamerfont{block body }{shape=\itshape}
  \setbeamerfont{block title}{series=\bfseries}
  \setbeamercolor{block title}{fg=MidnightBlue,bg=white}
  \setbeamercolor{block body}{fg=black, bg=gray!10}
}

% definition_fr
\theoremstyle{definition}
\newtheorem{definition_fr}{Définition}%[section]
\BeforeBeginEnvironment{definition_fr}{
  \setbeamerfont{block title}{series=\bfseries}
  \setbeamercolor{block title}{fg=MidnightBlue,bg=white}
  \setbeamercolor{block body}{fg=black, bg=gray!10}
}
\newtheorem*{definition_fr*}{Définition}
\BeforeBeginEnvironment{definition_fr*}{
  \setbeamerfont{block title}{series=\bfseries}
  \setbeamercolor{block title}{fg=MidnightBlue,bg=white}
  \setbeamercolor{block body}{fg=black, bg=gray!10}
}
% theorem_fr
\newtheorem{theorem_fr}{Théorème}%[section]
\BeforeBeginEnvironment{theorem_fr}{
  \setbeamerfont{block body}{shape=\itshape}
  \setbeamerfont{block title}{series=\bfseries, shape = \upshape}
  \setbeamercolor{block title}{fg=MidnightBlue,bg=white}
  \setbeamercolor{block body}{fg=black, bg=gray!10}
}
\newtheorem*{theorem_fr*}{Théorème}
\BeforeBeginEnvironment{theorem_fr*}{
  \setbeamerfont{block body}{shape=\itshape}
  \setbeamerfont{block title}{series=\bfseries, shape = \upshape}
  \setbeamercolor{block title}{fg=MidnightBlue,bg=white}
  \setbeamercolor{block body}{fg=black, bg=gray!10}
}

% remark_fr
\theoremstyle{remark}
\newtheorem{remark_fr}{Remarque}%[section]
\BeforeBeginEnvironment{remark_fr}{
  \setbeamerfont{block title}{series=\bfseries, shape=\itshape}
  \setbeamercolor{block title}{fg=MidnightBlue,bg=white}
  \setbeamercolor{block body}{fg=black, bg=gray!10}
}
\newtheorem*{remark_fr*}{Remarque}
\BeforeBeginEnvironment{remark_fr*}{
  \setbeamerfont{block title}{series=\bfseries, shape=\itshape}
  \setbeamercolor{block title}{fg=MidnightBlue,bg=white}
  \setbeamercolor{block body}{fg=black, bg=gray!10}
}








\usepackage[svgnames]{xcolor}
\usepackage{tikz}
\usetikzlibrary{shapes.geometric, arrows}
\usepackage{enumerate}   
\usepackage{multirow}
\usepackage{txfonts}
\usepackage{mathrsfs}
\usepackage{pgfplots}
\pgfplotsset{compat = newest}
\usetikzlibrary{positioning, arrows.meta}
\usepgfplotslibrary{fillbetween}
\newcommand{\A}{(0,0) ++(135:2) circle (2)}
\newcommand{\B}{(0,0) ++(45:2) circle (2)}
\DeclareMathOperator{\C}{C}
\DeclareMathOperator{\util}{u}
%\setbeamersize{text margin left=1.5em,text margin right=1.5em} 
%\setbeamersize{text margin left=1.2cm,text margin right=1.2cm} 
\setbeamersize{text margin left=1.5em,text margin right=1.5em} 
%\usepackage{xr}
%\externaldocument{Econometrie1_UGA_P2e}
  \usepackage{eso-pic}
%\newcommand\AtPagemyUpperLeft[1]{\AtPageLowerLeft{%
%\put(\LenToUnit{0.9\paperwidth},\LenToUnit{0.85\paperheight}){#1}}}
%\AddToShipoutPictureFG{
 % \AtPagemyUpperLeft{{\includegraphics[width=1.1cm,keepaspectratio]{logoUGA2020.pdf}}}
%}%

%\setbeamercolor{title}{fg=black}
%\setbeamercolor{frametitle}{fg=black}
%\setbeamercolor{section in head/foot}{fg=black}
%\setbeamercolor{author in head/foot}{bg=Brown}
%\setbeamercolor{date in head/foot}{fg=Brown}
\AtBeginSection[]
  {
    \ifnum \value{framenumber}>1
      \begin{frame}<beamer>
      \frametitle{Plan}
      \tableofcontents[currentsection]
      \end{frame}
    \else
    \fi
  }
\setbeamertemplate{section page}
{
    \begin{centering}
    \begin{beamercolorbox}[sep=11pt,center]{part title}
    \usebeamerfont{section title}\thesection.~\insertsection\par
    \end{beamercolorbox}
    \end{centering}
}

%\titlegraphic{\includegraphics[width=1cm]{logoUGA2020.pdf}}
\title[]{ \textbf{Introduction au méthodes statistiques}}
\subtitle{Cours 1}
\date{\today}
\author{Michal W. Urdanivia\inst{*}}
\institute{\inst{*}UGA, Facult\'e d'\'Economie, GAEL, \\e-mail: \href{mailto:michal.wong-urdanivia@univ-grenoble-alpes.fr}{michal.wong-urdanivia@univ-grenoble-alpes.fr}}

%\titlegraphic{\includegraphics[width=1cm]{logoUGA2020.pdf}
%}

\begin{document}
%%% TIKZ STUFF
\usetikzlibrary{positioning}
\usetikzlibrary{snakes}
\usetikzlibrary{calc}
\usetikzlibrary{arrows}
\usetikzlibrary{decorations.markings}
\usetikzlibrary{shapes.misc}
\usetikzlibrary{matrix,shapes,arrows,fit,tikzmark}
\usetikzlibrary{shapes}
\tikzset{   
        every picture/.style={remember picture,baseline},
        every node/.style={anchor=base,align=center,outer sep=1.5pt},
        every path/.style={thick},
        }
\newcommand\marktopleft[1]{%
    \tikz[overlay,remember picture] 
        \node (marker-#1-a) at (-.3em,.3em) {};%
}
\newcommand\markbottomright[2]{%
    \tikz[overlay,remember picture] 
        \node (marker-#1-b) at (0em,0em) {};%
}
\tikzstyle{every picture}+=[remember picture] 
\tikzstyle{mybox} =[draw=black, very thick, rectangle, inner sep=10pt, inner ysep=20pt]
\tikzstyle{fancytitle} =[draw=black,fill=red, text=white]
\tikzstyle{observed}=[draw,circle,fill=gray!50]



\begin{frame}
\titlepage
\end{frame}
\begin{frame}
 \tableofcontents
    \end{frame}
%\begin{frame}
%\frametitle{Contenu}
%\tableofcontents[pausesections, pausesubsections]
%\end{frame}

%\section{Qu'est-ce que l’économétrie ? A quoi (à qui) ça sert ?}
%\frame{\sectionpage}
%\begin{frame}
%  \tableofcontents  
%\end{frame}

\section{Présentation}
\frame{\sectionpage}

\begin{frame}[allowframebreaks]{Objectifs}
\begin{itemize}
    \item Une introduction formelle et solide à la méthodologie statistique.
    \item Une discussion de la pertinence/adéquation des méthodes selon les/aux applications.
\end{itemize}
\end{frame}

\begin{frame}[allowframebreaks]{Divers}
    \begin{itemize}
        \item \textbf{\underline{Prérequis:}} cours de probabilités, analyse, algèbre(calcul vectoriel, matriciel, orthogonalité).
        \item \textbf{\underline{Références:}} 
        \begin{enumerate}[-]
            \item Le matériel donné pour le cours d'abord.
            \item Pour une référence "encyclopédique"(même si le titre indique que c'est "concis") 
            vous pouvez consulter  \cite{Wasserman2004}.
            \item Des références supplémentaires pourront être indiquées au moment de traiter chaque thème.
        \end{enumerate}
        \framebreak
        \item \textbf{\underline{Travail:}} \begin{enumerate}
            \item Exercices(analytiques) pour s'approprier les méthodes et/ou traiter 
            de points non vus dans le cours.
            \item Applications/illustrations: elle seront données/corrigées sous forme de notebooks
             Python, mais vous pouvez utiliser le langage de votre préférence(R, Matlab, \ldots)
             \item Le travail sera donné environ toutes les deux semaines. 
             Les corrections seront mises en ligne et ne se feront pas en séance(au moins dans la totalité)
        \end{enumerate}
        \framebreak
        \item Site/dépôt du cours: 
        \medskip
        \url{https://github.com/MWUrda/Cours-Stats-UGA-M1-Eco-S1}
    
    \end{itemize}
    \end{frame}

\section{Introduction}
\frame{\sectionpage}

\begin{frame}[allowframebreaks]{Heuristique}
\begin{itemize}
    \item On considère \textbf{l'expérience statistique suivante:}.
    \begin{enumerate}[-]
        \item On lance une pièce de monnaie $n$ fois, et à chaque lancer est associé 
    son résultat qui vaut soit "pile", soit "face".
        \item On veut à partir de cette expérience mesurer la probabilité du résultat "pile", que nous notons $p$.
        \item Pour cela, \textbf{on estime} $p$ par le pourcentage de résultats de valeur "pile" par rapport au nombre total de lancers.
     \end{enumerate}
    \item \textbf{\underline{Question:}} comment garantir la validité de cette procédure?

\framebreak

    \item Formellement, la procédure consiste en:
    \begin{enumerate}[-]
    \item Pour $i=1, 2, \ldots, n$, on définit $P_i = 1$ si le $i$-ème lancé donne "pile" et $P_i=0$ sinon.
    \item L'estimateur de $p$ est la moyenne empirique:
    \begin{align*}
        \bar{P}_n &=\frac{1}{n}\sumin P_i.
    \end{align*}
    \end{enumerate}
    \item \textbf{\underline{Question:}} quelle est la précision de cet estimateur?
    \item Afin de répondre à ces questions on propose un modèle statistique qui fournit une description/approximation de l'expérience, et qui sera considérée comme suffisamment bonne
    (par rapport à certains critères) pour être utile.
    \framebreak

    \item Le modèle consiste en un ensemble  d'hypothèses sur les 
     observations $P_i, i=1, 2, \ldots, n$.
     \item  A partir de ces hypothèse on obtient des conclusions/résultats statistiques.

    \item Les hypothèses sont les suivantes:

\begin{enumerate}
    \item $P_i$ est une variable aléatoire(v.a.).
    \item La loi de $P_i$ est une loi de Bernouilli de paramètre $p$.
    \item Les $P_1, P_2, \ldots, P_n$ sont mutuellement indépendantes.
\end{enumerate}

\framebreak

     \item \textbf{\underline{Discussion:}}
    \begin{enumerate}
    \item Modéliser les observations comme de v.a. est une façon de modéliser 
    le fait que le modélisateur n'a qu'une information imparfaite sur 
    les conditions de l'expérience. Si cette information était parfaite on pourrait 
    prédire les résultats de l'expérience sans erreur.
    \item Comme $P_i\in\{0, 1\}$, la loi de $P_i$ est une loi de Bernouilli de paramètre $p$ 
    dès lors que la pièce est équilibrée.
    \medskip

    \ldots Pour une discussion intéressante vous pouvez consulter 
    \href{https://softmath.seas.harvard.edu/publication/probability-geometry-and-dynamics-in-the-toss-of-a-thick-coin/}
    {ici}
    \item Supposer l'indépendance est approprié lorsque 
    les conditions demeurent identiques entre les lancers. Par exemple, 
     il n'y a pas de processus d'apprentissage entre les lancers.
    \end{enumerate}
\end{itemize}
\end{frame}

\begin{frame}[allowframebreaks]{Deux résultats/outils importants}
    \begin{itemize}
        \item  Soit $X, X_1, X_2, \ldots, X_n$ des v.a. i.i.d., avec $\mu:=\Exp(X)$, 
    $\sigma^2 := \Vr(X)$.
    \begin{enumerate}
        \item \textbf{\underline{Lois des grands nombres(LGN):}}
        \begin{enumerate}[-]
            \item \textbf{\underline{Loi forte:}}
            \begin{align*}
                \bar{X}_n := \frac{1}{n}\sumin X_n &\limps\mu.
            \end{align*}
            \item \textbf{\underline{Loi faible:}}
            \begin{align*}
                \bar{X}_n := \frac{1}{n}\sumin X_n &\limp \mu.
            \end{align*}
        \end{enumerate}
        \framebreak
        \item \textbf{\underline{Théorème central limite(TCL):}}
        \begin{align*}
            \sqrt{n}\frac{\bar{X}_n - \mu}{\sigma} &\limd \mathcal{N}(0, 1).
        \end{align*}
        ou de manière équivalente
        \begin{align*}
            \sqrt{n}(\bar{X}_n - \mu) &\limd \mathcal{N}(0, \sigma^2).
        \end{align*}
    \end{enumerate}
\end{itemize}
\end{frame}

\begin{frame}[allowframebreaks]{Conséquences}
    \begin{itemize}
        \item  Les LGN appliquées à $\bar{P}_n$ impliquent:
    \begin{align*}
        \bar{P}_n \limp p, &\quad \textrm{ou} \quad  \bar{P}_n \limps p.
    \end{align*}
    \item Ainsi lorsque $n\rightarrow \infty$, $\bar{P_n}$ est un "bon" 
    estimateur(\emph{estimateur consistant}) de $p$.
        \item Le TCL affine ce résultat en quantifiant la précision de l'estimateur.
        \framebreak
        \item Notons: 
        \begin{enumerate}[-]
            \item $\Phi(x)$: fonction de répartition d'une v.a. de loi $\mathcal{N}(0, 1)$.
            \item $\Phi_n(x)$: fonction de répartition de $\sqrt{n}\frac{\bar{P}_n-p}{\sqrt{p(1-p)}}$.
        \end{enumerate}
        \item Par le TCL $\Phi_n(x)\approx \Phi(x)$ lorsque $n\rightarrow \infty$. Par conséquent, 
        pour tout $x> 0$,
        \begin{align*}
            \Prob\left(\abs{\bar{P}_n- p}\geq x\right)
            &\approx 2\left(1-\Phi\left(\frac{x\sqrt{n}}{\sqrt{p(1-p)}}\right)\right).
        \end{align*}
        \framebreak
        \item \textbf{\underline{Conséquences:}}
        \begin{enumerate}[-]
            \item Une approximation sur le degré de concentration de $\bar{P}_n$ autour de $\mu$;
            \item Pour $\alpha \in(0, 1)$, si $q_\alpha$ est le quantile d'ordre $(1-\alpha/2)$ 
            de la loi $\mathcal{N}(0, 1)$, alors avec une probabilité $\approx 1-\alpha$(pour $n\rightarrow \infty$): 
            \begin{align*}
                \bar{P}_n&\in\left[p - \frac{q_\alpha\sqrt{p(1-p)}}{\sqrt{n}};
                p + \frac{q_\alpha\sqrt{p(1-p)}}{\sqrt{n}}\right].
            \end{align*}
        \end{enumerate}
        \framebreak
        \item On remarque que quelque soit la valeur(inconnue) de $p$,\begin{align*}
            p(1-p)&\leq 1/4.
        \end{align*}
        \item Ce faisant on a grossièrement avec une probabilité d'au moins $1-\alpha$:\begin{align*}
            \bar{P}_n&\in\left[p - \frac{q_\alpha}{2\sqrt{n}};
             p + \frac{q_\alpha}{2\sqrt{n}}\right].
        \end{align*}
        \item Autrement dit lorsque $n\rightarrow \infty$, l'intervalle $\left[\bar{P}_n - \frac{q_\alpha}{2\sqrt{n}};
        \bar{P}_n + \frac{q_\alpha}{2\sqrt{n}}\right]$ contient $p$ avec une probabilité $\geq 1-\alpha$.
        \item Cet intervalle est appelé \emph{intervalle de confiance asymptotique} pour $p$.
        \item \textbf{\underline{Question:}} qu'en est-il pour $n$ pas "suffisamment" grand?
\end{itemize}
\end{frame}
\begin{frame}[allowframebreaks]{Un autre résultat/outil: inégalité de Hoeffding}
    \begin{itemize}
        \item On considère le cas i.i.d.
    \begin{enumerate}[-]
        \item Soit $n$ un entier positif et $X, X_1, X_2, \ldots, X_n$ des v.a. i.i.d. telles que 
        $X\in [a, b]$ p.s.($a<b$ sont deux nombres donnés). Soit $\mu :=\Exp(X)$. 
        \item Alors pour tout $\epsilon>0$:\begin{align*}
            \Prob\left(\abs{\bar{X}_n - \mu} \geq \epsilon\right)&\leq 2e^{-\frac{2n\epsilon^2}{(b-a)^2}}.
        \end{align*}
\end{enumerate}
\item \textbf{\underline{Conséquence:}}\begin{enumerate}[-]
    \item Pour $\alpha \in (0, 1)$, avec une probabilité $(1-\alpha)$:\begin{align*}
        \bar{P}_n-\sqrt{\frac{\log(2/\alpha)}{2n}}&\leq p \leq \bar{P}_n+\sqrt{\frac{\log(2/\alpha)}{2n}}.
    \end{align*}
    \item Ce résultat est valable pour $n$ "pas très grand".
\end{enumerate}
\end{itemize}
\end{frame}
       \section{Retour sur les différents types de convergence}
       \frame{\sectionpage}
       \begin{frame}
           [allowframebreaks]{Types de convergence}
           \begin{itemize}
               \item Soit $(T_n)_{n\geq 1}$ une suite de v.a. et $T$ une v.a.($T$ 
               peut être déterministe).\begin{enumerate}[-]
                   \item \textbf{\underline{Convergence presque sûre(p.s.):}}
                   \begin{align*}
                    T_n \limps T & \quad \textrm{ssi}\quad
                    \Prob\left[\left\{\omega: T_n(\omega)\underset{n\rightarrow\infty}{\longrightarrow} T(\omega)\right\}\right] = 1.
                   \end{align*}
                   \item \textbf{\underline{Convergence en probabilité:}} 
                   \begin{align*}
                    T_n \limp T & \quad \textrm{ssi}\quad \Prob\left(\abs{T_n-T}\geq \epsilon\right)
                    \underset{n\rightarrow\infty}{\longrightarrow} 0, \quad \forall \epsilon > 0.
                   \end{align*}
                   \framebreak
                   \item \textbf{\underline{Convergence en $L^p$($p\geq 1$):}}
                   \begin{align*}
                       T_n\limlp T & \quad \textrm{ssi}\quad \Exp\left(\abs{T_n-T}^p\right)
                       \underset{n\rightarrow\infty}{\longrightarrow} 0.
                   \end{align*} 
                   \item \textbf{\underline{Convergence en loi/distribution):}}
                    \begin{align*}
                    T_n\limd T & \quad \textrm{ssi}\quad \Prob(T_n\leq x) \underset{n\rightarrow\infty}{\longrightarrow} \Prob(T\leq x),
                     \end{align*} 
                   pour tout $x\in\R$ où la fonction de répartition/distribution de $T$ est
                   continue.
               \end{enumerate}
               \item \textbf{\underline{Remarque:}} les définition précédentes peuvent s'étendre aux vecteurs de v.a.(i.e., v.a. dans $\R^d$, pour $d\geq 2$).
           \end{itemize}
       \end{frame}
    \begin{frame}
        [allowframebreaks]{Caractérisations de la convergence en distribution}
        \begin{itemize}
            \item Les propositions suivantes sont équivalentes:\begin{enumerate}[(i)]
                \item $T_n\limd T$;
                \medskip
                \item $\Exp\left[f(T_n)\right] 
                \underset{n\rightarrow \infty}{\longrightarrow}\Exp\left[f(T)\right]$, 
                pour toutes les fonctions continues et bornées $f$;
                \medskip
                \item $\Exp\left(e^{ixT_n}\right)
                \underset{n\rightarrow \infty}{\longrightarrow} \Exp\left(e^{ixT}\right)$, 
                pour tout $x\in\R$.
            \end{enumerate}
        \end{itemize}
    \end{frame}
        \begin{frame}
            [allowframebreaks]{Propriétés importantes}\begin{itemize}
                \item Si $(T_n)_{n\geq 1}$ converge p.s., alors elle converge 
                aussi en probabilité, et les deux limites sont égales p.s.
                \item Si $(T_n)_{n\geq 1}$ converge en $L^p$ alors elle converge aussi en $L^q$ pour tout $q\leq p$ 
                et en probabilité, et les deux limites sont égales p.s.
                \item Si $f$ est une fonction continue: \begin{align*}
                    T_n\overset{p.s./p/d}{\longrightarrow} T &\Rightarrow 
                    f(T_n) \overset{p.s./p/d}{\longrightarrow} f(T).
                \end{align*}
            \end{itemize}
        \end{frame}
        \begin{frame}
            [allowframebreaks]{Limites et opérations}\begin{itemize}
                \item On peut additionner, multiplier, \ldots, les limites presque sûrement et en probabilité.
                Si $U_n \overset{p.s./p}{\longrightarrow} U$, et $V_n \overset{p.s./p}{\longrightarrow} V$, alors: 
                \begin{enumerate}[-]
                    \item $U_n + V_n \overset{p.s./p}{\longrightarrow} U+V$,
                    \item $U_nV_n \overset{p.s./p}{\longrightarrow} UV$,
                    \item Si en plus, $V\neq 0$ p.s., alors $\frac{U_n}{V_n}\overset{p.s./p}{\longrightarrow}\frac{U}{V}$.
                \end{enumerate}
                \item \textbf{\underline{Remarque:}} ces règles ne s'appliquent pas à la convergence en distribution sauf lorsque la pair $(U_n, V_n)$ 
                converge vers en distribution vers $(U, V)$.
            \end{itemize}
                 
        \end{frame}
        \begin{frame}
            [allowframebreaks]{Un autre exemple}
            \begin{itemize}
                \item On observe les temps entre les arrivées de personnes dans une file d'attente(par exemple d'un centre d'appel):
                $T_1, T_2, \ldots, T_n$.
                \item On suppose que ces temps sont: 
                \begin{enumerate}[-]
                    \item Mutuellement indépendants.
                    \item Représentés par des v.a de loi exponentielle de paramètre commun $\lambda>0$.
                \end{enumerate} 
                \item On veut estimer $\lambda$ à partir des observations.
                \framebreak
                \item \textbf{\underline{Discussion des hypothèses:}}    
                \begin{enumerate}[-]
                    \item Indépendance mutuelle: les personnes sont supposées ne pas être liées 
                    entre elles de sorte qu'elles ne décident pas leur arrivée en fonction de celle des autres.
                    \item $T_1, T_2, \ldots, T_n$ sont de v.a de loi exponentielle: absence de mémoire de la loi exponentielle;
                    \begin{align*}
                        \Prob(T_1 > t + s)&=\Prob(T_1>s), \quad \forall s, t \geq 0.
                    \end{align*}
                    \item Les lois exponentielles de $T_1, T_2, \ldots$ ont le même paramètre $\lambda$: 
                    comportement homogène dans la population.
                \end{enumerate}   
                \framebreak
                \item Densité de $T_1$: 
                \begin{align*}
                    f(t)&= \lambda e^{-\lambda t}, \quad \forall t \geq 0.
                \end{align*}
                \item $\Exp(T_1)=\frac{1}{\lambda}$.
                \item Et un estimateur naturel de $\frac{1}{\lambda}$ est: 
                \begin{align*}
                    \bar{T}_n &:=\frac{1}{n}\sumin T_i.
                \end{align*}   
                \item Un estimateur naturel de $\lambda$ est: 
                \begin{align*}
                    \hat{\lambda} &:=\frac{1}{\bar{T}_n}.
                \end{align*} 
                \framebreak
                \item Par les LGNs,
                \begin{align*}
                    \bar{T}_n &\overset{p.s./p}{\longrightarrow} \frac{1}{\lambda}.
                \end{align*}   
                \item Et donc: 
                \begin{align*}
                    \hat{\lambda}&\overset{p.s./p}{\longrightarrow} \lambda.
                \end{align*}     
                \item Par le TCL: 
                \begin{align*}
                    \sqrt{n}\left(\bar{T}_n - \frac{1}{\lambda}\right)&\limd \mathcal{N}(0, \lambda^{-2}).
                \end{align*}                               
                \item \textbf{\underline{Question:}} comment le résultat sur le TCL ci-dessus se traduit/transfère t-il 
                sur $\hat{\lambda}$? Comment obtenir un intervalle de confiance asymptotique pour $\lambda$?
            \end{itemize}
        \end{frame}
\begin{frame}
    [allowframebreaks]{La méthode Delta}
    \begin{itemize}
        \item Soit $(Z_n)_{n\geq 1}$ une suite de v.a. qui vérifient: 
        \begin{align*}
            \sqrt{n}\left(Z_n - \vartheta \right)&\limd \mathcal{N}(0, \sigma^2),
        \end{align*}
        pour $\vartheta \in \R$ et $\sigma^2 >0$(la suite $(Z_n)_{n\geq 1}$ est dite \emph{asymptotiquement normale autour de $\vartheta$}).
        \item Soit $g: \R \rightarrow \R$ continûment différentiable au point $\vartheta$. Alors: 
        \begin{enumerate}[-]
            \item $\left(g(Z_n)\right)_{n\geq 1}$ est aussi asymptotiquement normale.
            \item Plus précisément: 
            \begin{align*}
                \sqrt{n}\left(g(Z_n)-g(\vartheta)\right)&\limd \mathcal{N}(0, g^\prime(\vartheta)^2\sigma^2).
            \end{align*}
        \end{enumerate}
    \end{itemize}
\end{frame}
\begin{frame}
    [allowframebreaks]{Conséquences de la méthode Delta}
    \begin{itemize}
        \item $\sqrt{n}(\hat{\lambda}-\lambda) \limd \mathcal{N}(0,\lambda^2)$.
        \item Par conséquent, pour $\alpha\in(0,1)$ et $n$ suffisamment grand: 
        \begin{align*}
            \abs{\hat{\lambda}-\lambda} &\leq \frac{q_\alpha\lambda}{\sqrt{n}}.
        \end{align*}
        \item \textbf{\underline{Question:}} est-ce que 
        $\left[\hat{\lambda} - \frac{q_\alpha\lambda}{\sqrt{n}}; \hat{\lambda} 
        + \frac{q_\alpha\lambda}{\sqrt{n}}\right]$ peut être utilisé comme intervalle de confiance asymptotique pour $\lambda$?
        \item La réponse est non car il dépend de $\lambda$.
        \framebreak
        \item Deux solutions sont possibles.
        \begin{enumerate}
        \item \textbf{\underline{Solution sous dépendance:}}
        \begin{enumerate}[-]
        \item Nous avons:
            \begin{align*}
                \abs{\hat{\lambda}-\lambda} \leq \frac{q_\alpha\lambda}{\sqrt{n}}&\Leftrightarrow 
                \lambda\left(1- \frac{q_\alpha}{\sqrt{n}}\right)\leq \hat{\lambda}
                \leq \lambda\left(1+ \frac{q_\alpha}{\sqrt{n}}\right)\\
                &\Leftrightarrow \hat{\lambda}\left(1 + \frac{q_\alpha}{\sqrt{n}}\right)^{-1}
                \leq \lambda \leq \hat{\lambda}\left(1 - \frac{q_\alpha}{\sqrt{n}}\right)^{-1}
            \end{align*}
        \item Et l'intervalle de confiance asymptotique pour lambda est: 
        \[
            \left[\hat{\lambda}\left(1 + \frac{q_\alpha}{\sqrt{n}}\right)^{-1}; \hat{\lambda}\left(1 - \frac{q_\alpha}{\sqrt{n}}\right)^{-1}\right]
        \]
        \end{enumerate}
        \item Solution systématique par le \textbf{théorème de Slutsky}.
    \end{enumerate}
    \end{itemize}
\end{frame}
\begin{frame}
    [allowframebreaks]{Théorème de Slutsky}
    \begin{itemize}
        \item Soit $(X_n)_{n\geq 1}, (Y_n)_{n\geq 1}$ deux suites de v.a. telles que: 
        \begin{enumerate}[(i)]
            \item $X_n \limd X$;
            \item $Y_n \limp c$, 
        \end{enumerate}
        où $X$ est une v.a. et $c$ un nombre réel donné. 
        \item Alors: \begin{align*}
            (X_n, Y_n) &\limd (X, c).
        \end{align*}
        \item En particulier: \begin{align*}
            X_n + Y_n &\limd X + c,\\
            X_nY_n&\limd cX,\\
            \cdots
        \end{align*}
    \end{itemize}
\end{frame}
\begin{frame}
    [allowframebreaks]{Conséquences du théorème de Slutsky}
    \begin{itemize}
        \item Par la méthode Delta, on sait que: \begin{align*}
            \sqrt{n}\frac{\hat{\lambda} - \lambda}{\lambda} &\limd \mathcal{N}(0,1).
        \end{align*}
        \item Par la loi faible des grands nombres:\begin{align*}
            \hat{\lambda} &\limp \lambda.
        \end{align*}
        \item Et donc par le théorème de Slutsky: \begin{align*}
            \sqrt{n}\frac{\hat{\lambda}-\lambda}{\hat{\lambda}}&\limd \mathcal{N}(0,1).
        \end{align*}
        \item Un autre intervalle de confiance asymptotique pour $\lambda$ est: \begin{align*}
            \left[\hat{\lambda}- \frac{q_\alpha\hat{\lambda}}{\sqrt{n}}; 
            \hat{\lambda}+ \frac{q_\alpha\hat{\lambda}}{\sqrt{n}}\right].
        \end{align*}

        \framebreak
        
        \textbf{\underline{Remarque:}}\begin{enumerate}[-]
            \item Dans le premier example(lancers de pièce avec résultat pile/face), nous avons 
            utilisé une solution sous dépendance: "$p(1-p)\leq 1/4$".
            \item Avec le théorème de Slutsky nous obtenons l'intervalle de confiance asymptotique:\begin{align*}
                \left[\hat{P}_n - \frac{q_\alpha\sqrt{\hat{P}_n(1-\hat{P}_n)}}{\sqrt{n}}; 
                \hat{P}_n + \frac{q_\alpha\sqrt{\hat{P}_n(1-\hat{P}_n)}}{\sqrt{n}}
                \right].
            \end{align*}
        \end{enumerate}

    \end{itemize}
    
\end{frame}
\begin{frame}[allowframebreaks]{Références}
\bibliographystyle{jpe}
\bibliography{../Biblio}

\end{frame}
\end{document}
