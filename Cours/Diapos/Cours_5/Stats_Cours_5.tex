%\documentclass[ignorenonframetext, compress, 9pt, xcolor=svgnames]{beamer} 
\documentclass[notes, ignorenonframetext, compress, 11pt, xcolor=svgnames, aspectratio=169]{beamer} 
\usepackage{pgfpages}
\usepackage{pdfpages}
% These slides also contain speaker notes. You can print just the slides,
% just the notes, or both, depending on the setting below. Comment out the want
% you want.
\setbeameroption{hide notes} % Only slide
%\setbeameroption{show only notes} % Only notes
%\setbeameroption{show notes on second screen=right} % Both
\usepackage{amsmath}
\usepackage{amsfonts}
\usepackage{amssymb}
\setbeamercolor{frametitle}{fg=MidnightBlue}

\setbeamercolor{sectionpage title}{bg=MidnightBlue}
\setbeamertemplate{frametitle}[default][center]
%\setbeamertemplate{frametitle}{\color{MidnightBlue}\centering\bfseries\insertframetitle\par\vskip-6pt}
\setbeamerfont{frametitle}{series=\bfseries}
\setbeamerfont{title}{series=\bfseries}
\setbeamerfont{sectionpage}{series=\bfseries}
%\setbeamercolor{section in head/foot}{bg=MidnightBlueBlue}
%\setbeamercolor{author in head/foot}{bg=DarkBlue}
\setbeamercolor{author in head/foot}{fg=MidnightBlue}
%\setbeamercolor{title in head/foot}{bg=White}
\setbeamercolor{title in head/foot}{fg=MidnightBlue}
\setbeamercolor{title}{fg=MidnightBlue}
%\setbeamercolor{date in head/foot}{fg=Brown}
%\setbeamercolor{alerted text}{fg=DarkBlue}
%\usecolortheme[named=DarkBlue]{structure} 
%\usepackage{bbm}
%\usepackage{bbold}
\usepackage{eurosym}
\usepackage{graphicx}
%\usepackage{epstopdf}
\usepackage{hyperref}
\hypersetup{
  colorlinks   = true, %Colours links instead of ugly boxes
  urlcolor     = gray, %Colour for external hyperlinks
  linkcolor    = MidnightBlue, %Colour of internal links
  citecolor   = DarkRed %Colour of citations
}
\usepackage{multirow}
\usepackage{xspace}
\usepackage{listings}
\usepackage{natbib}
%\usepackage[sort&compress,comma,super]{natbib}
\def\newblock{} % To avoid a compilation error about a function \newblock undefined
\usepackage{bibentry}
\usepackage{booktabs}
\usepackage{dcolumn}
\usepackage[greek,frenchb]{babel}
\usepackage[babel=true,kerning=true]{microtype}
\usepackage[utf8]{inputenc}
\usepackage[T1]{fontenc}
\usepackage{natbib}
\renewcommand{\cite}{\citet}
\usepackage{longtable}
\usepackage{eso-pic}

\usepackage{xcolor}
 \colorlet{linkequation}{DarkRed} 
 \newcommand*{\SavedEqref}{}
 \let\SavedEqref\eqref 
\renewcommand*{\eqref}[1]{%
\begingroup \hypersetup{
      linkcolor=linkequation,
linkbordercolor=linkequation, }%
\SavedEqref{#1}%
 \endgroup
}

\newcommand*{\refeq}[1]{%
 \begingroup
\hypersetup{ 
linkcolor=linkequation, 
linkbordercolor=linkequation,
}%
\ref{#1}%
 \endgroup
}

\setbeamertemplate{caption}[numbered]
\setbeamertemplate{theorem}[ams style]
\setbeamertemplate{theorems}[numbered]
%\usefonttheme{serif}
%\usecolortheme{beaver}
%\usetheme{Hannover}
%\usetheme{CambridgeUS}
%\usetheme{Madrid}
%\usecolortheme{whale}
%\usetheme{Warsaw}
%\usetheme{Luebeck}
%\usetheme{Montpellier}
%\usetheme{Berlin}
%\setbeamercolor{titlelike}{parent=structure}
%\setbeamertemplate{headline}[default]
%\setbeamertemplate{footline}[default]
%\setbeamertemplate{footline}[Malmoe]
%\setbeamercovered{transparent}
%\setbeamercovered{invisible}
%\usecolortheme{crane}
%\usecolortheme{dolphin}
%\usepackage{pxfonts}
%\usepackage{isomath}
%\usepackage{mathpazo}
%\usepackage{arev} %     (Arev/Vera Sans)
%\usepackage{eulervm} %_   (Euler Math)
%\usepackage{fixmath} %  (Computer Modern)
%\usepackage{hvmath} %_   (HV-Math/Helvetica)
%\usepackage{tmmath} %_   (TM-Math/Times)
%\usepackage{tgheros}
%\usepackage{cmbright}
%\usepackage{ccfonts} \usepackage[T1]{fontenc}
%\usepackage[garamond]{mathdesign}

%\usepackage{color}
%\usepackage{ulem}

%\usepackage[math]{kurier}
%\usepackage[no-math]{fontspec}
%\setmainfont{Fontin Sans}
%\setsansfont{Fontin Sans}
%\setbeamerfont{frametitle}{size=\LARGE,series=\bfseries}
%%%add 19022021
\usepackage{enumerate}    
\usepackage{dcolumn}
\usepackage{verbatim}
\newcolumntype{d}[0]{D{.}{.}{5}}
%\setbeamertemplate{note page}{\pagecolor{yellow!5}\insertnote}
%\usetikzlibrary{positioning}
%\usetikzlibrary{snakes}
%\usetikzlibrary{calc}
%\usetikzlibrary{arrows}
%\usetikzlibrary{decorations.markings}
%\usetikzlibrary{shapes.misc}
%\usetikzlibrary{matrix,shapes,arrows,fit,tikzmark}
%%%
% suppress navigation bar
\beamertemplatenavigationsymbolsempty
%\usetheme{bunsenMod}
%\setbeamercovered{transparent}
%\setbeamertemplate{items}[circle]
%\usecolortheme[named=CadetBlue]{structure}
%\usecolortheme[RGB={225,64,5}]{structure}
%\definecolor{burntRed}{RGB}{225,64,5}
%\setbeamercolor{alerted text}{fg=burntRed} 
%\usecolortheme[RGB={0,40,110}]{structure}
%\hypersetup{linkcolor=burntRed}
%\hypersetup{urlcolor=burntRed}
%\hypersetup{filecolor=burntRed}
%\hypersetup{citecolor=burntRed}

%\usetheme{bunsenMod}
%\setbeamercovered{transparent}
%\setbeamertemplate{items}[circle]
%\usecolortheme[named=CadetBlue]{structure}
%\usecolortheme[RGB={225,64,5}]{structure}
%\definecolor{burntRed}{RGB}{225,64,5}
%\setbeamercolor{alerted text}{fg=burntRed} 
%\usecolortheme[RGB={0,40,110}]{structure}
%\hypersetup{linkcolor=burntRed}
%\hypersetup{urlcolor=burntRed}
%\hypersetup{filecolor=burntRed}
%\hypersetup{citecolor=burntRed}

%\AtBeginSection[] % Do nothing for \section*
%{ \frame{\sectionpage} }
%\setbeamertemplate{frametitle continuation}{}
\newtheorem{lemme}{Lemme}[section]
%\newtheorem{remarque}{Remarque}
\newcommand{\argmax}{\operatornamewithlimits{arg\,max}}
\newcommand{\argmin}{\operatornamewithlimits{arg\,min}}
\def\inprobLOW{\rightarrow_p}
\def\inprobHIGH{\,{\buildrel p \over \rightarrow}\,} 
\def\inprob{\,{\inprobHIGH}\,} 
\def\indist{\,{\buildrel d \over \rightarrow}\,} 
\def\sima{\,{\buildrel a \over \sim}\,} 
\def\F{\mathbb{F}}
\def\R{\mathbb{R}}
\def\N{\mathbb{N}}
\newcommand{\gmatrix}[1]{\begin{pmatrix} {#1}_{11} & \cdots &
    {#1}_{1n} \\ \vdots & \ddots & \vdots \\ {#1}_{m1} & \cdots &
    {#1}_{mn} \end{pmatrix}}
\newcommand{\iprod}[2]{\left\langle {#1} , {#2} \right\rangle}
\newcommand{\norm}[1]{\left\Vert {#1} \right\Vert}
\newcommand{\abs}[1]{\left\vert {#1} \right\vert}
\renewcommand{\det}{\mathrm{det}}
\newcommand{\rank}{\mathrm{rank}}
\newcommand{\spn}{\mathrm{span}}
\newcommand{\row}{\mathrm{Row}}
\newcommand{\col}{\mathrm{Col}}
\renewcommand{\dim}{\mathrm{dim}}
\newcommand{\prefeq}{\succeq}
\newcommand{\pref}{\succ}
\newcommand{\seq}[1]{\{{#1}_n \}_{n=1}^\infty }
\renewcommand{\to}{{\rightarrow}}
\renewcommand{\L}{{\mathcal{L}}}
\newcommand{\Er}{\mathrm{E}}
\renewcommand{\Pr}{\mathrm{P}}
%\newcommand{\Var}{\mathrm{Var}}
%\newcommand{\Cov}{\mathrm{Cov}}
%\newcommand{\corr}{\mathrm{Corr}}
%\newcommand{\Var}{\mathrm{Var}}
\newcommand{\bias}{\mathrm{Bias}}
\newcommand{\mse}{\mathrm{MSE}}
\providecommand{\Pred}{\mathcal{P}}
\providecommand{\plim}{\operatornamewithlimits{plim}}
\providecommand{\avg}{\frac{1}{n} \underset{i=1}{\overset{n}{\sum}}}
\providecommand{\sumin}{{\sum_{i=1}^n}}
\providecommand{\limp}{\overset{p}{\rightarrow}}
%\providecommand{\limp}{\underset{n \rightarrow \infty}{\overset{p}{\longrightarrow}}}
%\providecommand{\limp}{\underset{n \rightarrow \infty}{\overset{p}{\longrightarrow}}}
%\providecommand{\limp}{\overset{p}{\longrightarrow}}
%\providecommand{\limd}{\underset{n \rightarrow \infty}{\overset{d}{\longrightarrow}}}
\providecommand{\limd}{\overset{d}{\rightarrow}}
\providecommand{\limps}{\overset{p.s.}{\rightarrow}}
\providecommand{\limlp}{\overset{L^p}{\rightarrow}}
\def\independenT#1#2{\mathrel{\setbox0\hbox{$#1#2$}%
    \copy0\kern-\wd0\mkern4mu\box0}} 
\newcommand\indep{\protect\mathpalette{\protect\independenT}{\perp}}


\lstset{language=R}
\lstset{keywordstyle=\color[rgb]{0,0,1},                                        % keywords
        commentstyle=\color[rgb]{0.133,0.545,0.133},    % comments
        stringstyle=\color[rgb]{0.627,0.126,0.941}      % strings
}       
\lstset{
  showstringspaces=false,       % not emphasize spaces in strings 
  columns=fixed,
  numbersep=3mm, numbers=left, numberstyle=\tiny,       % number style
  frame=none,
  framexleftmargin=5mm, xleftmargin=5mm         % tweak margins
}
\makeatletter
%\setbeamertemplate{frametitle continuation}{\gdef\beamer@frametitle{}}
\setbeamertemplate{frametitle continuation}{\frametitle{}}
%\setbeamertemplate{frametitle continuation}{\insertcontinuationcount}
\makeatother

\theoremstyle{remark}
\newtheorem{interpretation}{Interprétation}
\newtheorem*{interpretation*}{Interprétation}

\theoremstyle{remark}
\newtheorem{remarque}{Remarque}%[section]
\newtheorem*{remarque*}{Remarque}
\usepackage[framemethod=TikZ]{mdframed} 
\usepackage{showexpl}
%\newtheorem{step}{Step}[section]
%\newtheorem{rem}{Comment}[section]
%\newtheorem{ex}{Example}[section]
%\newtheorem{hist}{History}[section]
%\newtheorem*{ex*}{Example}
\theoremstyle{plain}
\newtheorem{propriete}{Propri\'et\'e}
\renewcommand{\thepropriete}{P\arabic{propriete}}
%\theoremstyle{definition}
%\newtheorem{definition}{Définition}%[section]
\theoremstyle{remark}
\newtheorem{exemple}{Exemple}
\newtheorem*{exemple*}{Exemple}

\newtheorem{theoreme}{Théorème}
\newtheorem{proposition}{Proposition}
%\newtheorem{propriete}{Propri\'et\'e}
\newtheorem{corollaire}{Corollaire}
%\newtheorem{exemple}{Exemple}
\newtheorem{assumption}{Assumption}
\renewcommand{\theassumption}{A\arabic{assumption}}
\newtheorem{hypothese}{Hypothèse}
\renewcommand{\thehypothese}{H\arabic{hypothese}}
\theoremstyle{definition}

%\newtheorem{definitionx}{D\'efinition}%[section]
%\newenvironment{definition}
 %{\pushQED{\qed}\renewcommand{\qedsymbol}{$\triangle$}\definitionx}
 %{\popQED\enddefinitionx}

\newtheorem{condition}{Condition}
\renewcommand{\thecondition}{C\arabic{condition}}
%\newcommand{\Var}{\mathbb{V}}
%\newcommand{\Var}{\mathbf{Var}}
%\newcommand{\Exp}{\mathbf{E}}
%\providecommand{\Vr}{\mathrm{Var}}
%\renewcommand{\Er}{\mathbb{E}}
%\newcommand{\LP}{\mathcal{LP}}
%\providecommand{\Id}{\mathbf{I}}
%\providecommand{\Rang}{\mathrm{Rang}}
%\providecommand{\Trace}{\mathrm{Trace}}
%\newcommand{\Cov}{\mathbf{Cov}}
%\newcommand{\Cov}{\mathbb{C}\mathrm{ov}}
\providecommand{\Id}{\mathbf{I}}
\providecommand{\Ind}{\mathbf{1}}
\providecommand{\uvec}{\mathbf{1}}
\providecommand{\vecOnes}{\mathbf{1}}
\DeclareMathOperator{\indfun}{\mathbf{1}}
\DeclareMathOperator{\Exp}{E}
\DeclareMathOperator{\Expn}{\mathbb{E}_n}
\DeclareMathOperator{\Var}{Var}
\DeclareMathOperator{\Vr}{V}
\DeclareMathOperator{\Cov}{Cov}
\DeclareMathOperator{\corr}{corr}
\DeclareMathOperator{\perps}{\perp_s}
%\DeclareMathOperator{\Prob}{Pr}
\DeclareMathOperator{\Prob}{P}
\DeclareMathOperator{\prob}{p}
\DeclareMathOperator{\loss}{L}
\providecommand{\Corr}{\mathrm{Corr}}
\providecommand{\Diag}{\mathrm{Diag}}
\providecommand{\reg}{\mathrm{r}}
\providecommand{\Likelihood}{\mathrm{L}}
\renewcommand{\Pr}{{\mathbb{P}}}
\providecommand{\set}[1]{\left\{#1\right\}}
\providecommand{\uvec}{\mathbf{1}}
\providecommand{\Rang}{\mathrm{Rang}}
\providecommand{\Trace}{\mathrm{Trace}}
\providecommand{\Tr}{\mathrm{Tr}}
\providecommand{\CI}{\mathrm{CI}}
\providecommand{\asyvar}{\mathrm{AsyVar}}
\DeclareMathOperator{\Supp}{Supp}
\newcommand{\inputslide}[2]{{
    \usebackgroundtemplate{
     \includegraphics[page={#2},width=0.90\textwidth,keepaspectratio=true]
      %\includegraphics[page={#2},width=\paperwidth,keepaspectratio=true]
      {{#1}}}
    \frame[plain]{}
  }}
\newcommand\pperp{\perp\!\!\!\perp}
\newcommand\independent{\protect\mathpalette{\protect\independenT}{\perp}}
\def\independenT#1#2{\mathrel{\rlap{$#1#2$}\mkern2mu{#1#2}}}
\usepackage{bbm}
\providecommand{\Ind}{\mathbf{1}}
\newcommand{\sumjsi}{\underset{i<j}{{\sum}}}
\newcommand{\prodjsi}{\underset{i<j}{{\prod}}}
\newcommand{\sumisj}{\underset{j<i}{{\sum}}}
\newcommand{\prodisj}{\underset{j<i}{{\prod}}}
\newcommand{\sumobs}{\underset{i=1}{\overset{n}{\sum}}}
\newcommand{\sumi}{\underset{i=1}{\overset{n}{\sum}}}
\newcommand{\prodi}{\underset{i=1}{\overset{n}{\prod}}}
\newcommand{\prodobs}{\underset{i=1}{\overset{n}{\prod}}}
\newcommand{\simiid}{{\overset{i.i.d.}{\sim}}}
%\newcommand{\sumobs}{\sum_{i=1}^N}
%\newcommand{\prodobs}{\prod_{i=1}^N}
%\newcommand{\sumjsi}{\sum_{i<j}}
%\newcommand{\prodjsi}{\prod_{i<j}}
%\newcommand{\sumisj}{\sum_{j<i}}
%\newcommand{\prodisj}{\sum_{j<i}}

%\usepackage{appendixnumberbeamer}
\setbeamertemplate{footline}[frame number]
\setbeamertemplate{section in toc}[sections numbered]
\setbeamertemplate{subsection in toc}[subsections numbered]
\setbeamertemplate{subsubsection in toc}[subsubsections numbered]

%\makeatother
%\setbeamertemplate{footline}
%{
%    \leavevmode%
%    \hbox{%
%        \begin{beamercolorbox}[wd=.333333\paperwidth,ht=2.25ex,dp=1ex,center]{author in head/foot}%
%            \usebeamerfont{author in head/foot}\insertshortauthor
%        \end{beamercolorbox}%
%        \begin{beamercolorbox}[wd=.333333\paperwidth,ht=2.25ex,dp=1ex,center]{title in head/foot}%
%            \usebeamerfont{title in head/foot}\insertshorttitle
%        \end{beamercolorbox}%
%        \begin{beamercolorbox}[wd=.333333\paperwidth,ht=2.25ex,dp=1ex,right]{date in head/foot}%
%            \usebeamerfont{date in head/foot}\insertshortdate{}\hspace*{2em}
%            \insertframenumber{} / \inserttotalframenumber\hspace*{2ex} 
%        \end{beamercolorbox}}%
%       \vskip0pt%
 %   }
%   \makeatother
%\setbeamertemplate{navigation symbols}{}
\setbeamertemplate{itemize items}[ball]
%\setbeamertemplate{itemize items}{-}
%\newenvironment{wideitemize}{\itemize\addtolength{\itemsep}{10pt}}{\enditemize}
% \usepackage{eso-pic}
%\newcommand\AtPagemyUpperLeft[1]{\AtPageLowerLeft{%
%\put(\LenToUnit{0.9\paperwidth},\LenToUnit{0.9\paperheight}){#1}}}
%\AddToShipoutPictureFG{
%  \AtPagemyUpperLeft{{\includegraphics[width=1.1cm,keepaspectratio]{../logo-uga.png}}}
%}%
\def\figheight{3in}
\def\figwidth{4in}

%%Commands from Econometric Theory(Slides) by J. Stachurski.

\newcommand{\boldx}{ {\mathbf x} }
\newcommand{\boldu}{ {\mathbf u} }
\newcommand{\boldv}{ {\mathbf v} }
\newcommand{\boldw}{ {\mathbf w} }
\newcommand{\boldy}{ {\mathbf y} }
\newcommand{\boldb}{ {\mathbf b} }
\newcommand{\bolda}{ {\mathbf a} }
\newcommand{\boldc}{ {\mathbf c} }
\newcommand{\boldd}{ {\mathbf d} }
\newcommand{\boldi}{ {\mathbf i} }
\newcommand{\bolde}{ {\mathbf e} }
\newcommand{\boldp}{ {\mathbf p} }
\newcommand{\boldq}{ {\mathbf q} }
\newcommand{\bolds}{ {\mathbf s} }
\newcommand{\boldt}{ {\mathbf t} }
\newcommand{\boldz}{ {\mathbf z} }
\newcommand{\boldr}{ {\mathbf r} }

\newcommand{\boldzero}{ {\mathbf 0} }
\newcommand{\boldone}{ {\mathbf 1} }

\newcommand{\boldalpha}{ {\boldsymbol \alpha} }
\newcommand{\boldbeta}{ {\boldsymbol \beta} }
\newcommand{\boldgamma}{ {\boldsymbol \gamma} }
\newcommand{\boldtheta}{ {\boldsymbol \theta} }
\newcommand{\boldxi}{ {\boldsymbol \xi} }
\newcommand{\boldtau}{ {\boldsymbol \tau} }
\newcommand{\boldepsilon}{ {\boldsymbol \epsilon} }
\newcommand{\boldmu}{ {\boldsymbol \mu} }
\newcommand{\boldSigma}{ {\boldsymbol \Sigma} }
\newcommand{\boldOmega}{ {\boldsymbol \Omega} }
\newcommand{\boldPhi}{ {\boldsymbol \Phi} }
\newcommand{\boldLambda}{ {\boldsymbol \Lambda} }
\newcommand{\boldphi}{ {\boldsymbol \phi} }

\newcommand{\Sigmax}{ {\boldsymbol \Sigma_{\boldx}}}
\newcommand{\Sigmau}{ {\boldsymbol \Sigma_{\boldu}}}
\newcommand{\Sigmaxinv}{ {\boldsymbol \Sigma_{\boldx}^{-1}}}
\newcommand{\Sigmav}{ {\boldsymbol \Sigma_{\boldv \boldv}}}

\newcommand{\hboldx}{ \hat {\mathbf x} }
\newcommand{\hboldy}{ \hat {\mathbf y} }
\newcommand{\hboldb}{ \hat {\mathbf b} }
\newcommand{\hboldu}{ \hat {\mathbf u} }
\newcommand{\hboldtheta}{ \hat {\boldsymbol \theta} }
\newcommand{\hboldtau}{ \hat {\boldsymbol \tau} }
\newcommand{\hboldmu}{ \hat {\boldsymbol \mu} }
\newcommand{\hboldbeta}{ \hat {\boldsymbol \beta} }
\newcommand{\hboldgamma}{ \hat {\boldsymbol \gamma} }
\newcommand{\hboldSigma}{ \hat {\boldsymbol \Sigma} }

\newcommand{\boldA}{\mathbf A}
\newcommand{\boldB}{\mathbf B}
\newcommand{\boldC}{\mathbf C}
\newcommand{\boldD}{\mathbf D}
\newcommand{\boldI}{\mathbf I}
\newcommand{\boldL}{\mathbf L}
\newcommand{\boldM}{\mathbf M}
\newcommand{\boldP}{\mathbf P}
\newcommand{\boldQ}{\mathbf Q}
\newcommand{\boldR}{\mathbf R}
\newcommand{\boldX}{\mathbf X}
\newcommand{\boldU}{\mathbf U}
\newcommand{\boldV}{\mathbf V}
\newcommand{\boldW}{\mathbf W}
\newcommand{\boldY}{\mathbf Y}
\newcommand{\boldZ}{\mathbf Z}

\newcommand{\bSigmaX}{ {\boldsymbol \Sigma_{\hboldbeta}} }
\newcommand{\hbSigmaX}{ \mathbf{\hat \Sigma_{\hboldbeta}} }
\newcommand{\betahat}{\hat{\beta}}
\newcommand{\gammahat}{\hat{\gamma}}
\newcommand{\Uhat}{\hat{U}}
\newcommand{\Vhat}{\hat{V}}
\newcommand{\epsilonhat}{\hat{\epsilon}}
\newcommand{\sigmahat}{\hat{\sigma}}
\newcommand{\Sigmahat}{\hat{\Sigma}}
\newcommand{\Gammahat}{\hat{\Gamma}}

\newcommand{\RR}{\mathbbm R}
\newcommand{\CC}{\mathbbm C}
\newcommand{\NN}{\mathbbm N}
\newcommand{\PP}{\mathbbm P}
\newcommand{\EE}{\mathbbm E \nobreak\hspace{.1em}}
\newcommand{\EEP}{\mathbbm E_P \nobreak\hspace{.1em}}
\newcommand{\ZZ}{\mathbbm Z}
\newcommand{\QQ}{\mathbbm Q}


\newcommand{\XX}{\mathcal X}

\newcommand{\aA}{\mathcal A}
\newcommand{\fF}{\mathscr F}
\newcommand{\bB}{\mathscr B}
\newcommand{\iI}{\mathscr I}
\newcommand{\rR}{\mathscr R}
\newcommand{\dD}{\mathcal D}
\newcommand{\lL}{\mathcal L}
\newcommand{\llL}{\mathcal{H}_{\ell}}
\newcommand{\gG}{\mathcal G}
\newcommand{\hH}{\mathcal H}
\newcommand{\nN}{\textrm{\sc n}}
\newcommand{\lN}{\textrm{\sc ln}}
\newcommand{\pP}{\mathscr P}
\newcommand{\qQ}{\mathscr Q}
\newcommand{\xX}{\mathcal X}

\newcommand{\ddD}{\mathscr D}


%\newcommand{\R}{{\texttt R}}
\newcommand{\risk}{\mathcal R}
\newcommand{\Remp}{R_{{\rm emp}}}

\newcommand*\diff{\mathop{}\!\mathrm{d}}
\newcommand{\ess}{ \textrm{{\sc ess}} }
\newcommand{\tss}{ \textrm{{\sc tss}} }
\newcommand{\rss}{ \textrm{{\sc rss}} }
\newcommand{\rssr}{ \textrm{{\sc rssr}} }
\newcommand{\ussr}{ \textrm{{\sc ussr}} }
\newcommand{\zdata}{\mathbf{z}_{\mathcal D}}
\newcommand{\Pdata}{P_{\mathcal D}}
\newcommand{\Pdatatheta}{P^{\mathcal D}_{\theta}}
\newcommand{\Zdata}{Z_{\mathcal D}}


\newcommand{\e}[1]{\mathbbm{E}[{#1}]}
\newcommand{\p}[1]{\mathbbm{P}({#1})}
% definition
\BeforeBeginEnvironment{definition}{
  \setbeamerfont{block title}{series=\bfseries}
  \setbeamercolor{block title}{fg=MidnightBlue,bg=white}
  \setbeamercolor{block body}{fg=black, bg=gray!10}
}
\newtheorem*{definition*}{Definition}
\BeforeBeginEnvironment{definition*}{
  \setbeamerfont{block title}{series=\bfseries}
  \setbeamercolor{block title}{fg=MidnightBlue,bg=white}
  \setbeamercolor{block body}{fg=black, bg=gray!10}
}

% theorem
\BeforeBeginEnvironment{theorem}{
  \setbeamerfont{block body}{shape=\itshape}
  \setbeamerfont{block title}{series=\bfseries}
  \setbeamercolor{block title}{fg=MidnightBlue,bg=white}
  \setbeamercolor{block body}{fg=black, bg=gray!10}
}
\newtheorem*{theorem*}{Theorem}
\BeforeBeginEnvironment{theorem*}{
  \setbeamerfont{block body }{shape=\itshape}
  \setbeamerfont{block title}{series=\bfseries}
  \setbeamercolor{block title}{fg=MidnightBlue,bg=white}
  \setbeamercolor{block body}{fg=black, bg=gray!10}
}

% definition_fr
\theoremstyle{definition}
\newtheorem{definition_fr}{Définition}%[section]
\BeforeBeginEnvironment{definition_fr}{
  \setbeamerfont{block title}{series=\bfseries}
  \setbeamercolor{block title}{fg=MidnightBlue,bg=white}
  \setbeamercolor{block body}{fg=black, bg=gray!10}
}
\newtheorem*{definition_fr*}{Définition}
\BeforeBeginEnvironment{definition_fr*}{
  \setbeamerfont{block title}{series=\bfseries}
  \setbeamercolor{block title}{fg=MidnightBlue,bg=white}
  \setbeamercolor{block body}{fg=black, bg=gray!10}
}
% theorem_fr
\newtheorem{theorem_fr}{Théorème}%[section]
\BeforeBeginEnvironment{theorem_fr}{
  \setbeamerfont{block body}{shape=\itshape}
  \setbeamerfont{block title}{series=\bfseries, shape = \upshape}
  \setbeamercolor{block title}{fg=MidnightBlue,bg=white}
  \setbeamercolor{block body}{fg=black, bg=gray!10}
}
\newtheorem*{theorem_fr*}{Théorème}
\BeforeBeginEnvironment{theorem_fr*}{
  \setbeamerfont{block body}{shape=\itshape}
  \setbeamerfont{block title}{series=\bfseries, shape = \upshape}
  \setbeamercolor{block title}{fg=MidnightBlue,bg=white}
  \setbeamercolor{block body}{fg=black, bg=gray!10}
}

% remark_fr
\theoremstyle{remark}
\newtheorem{remark_fr}{Remarque}%[section]
\BeforeBeginEnvironment{remark_fr}{
  \setbeamerfont{block title}{series=\bfseries, shape=\itshape}
  \setbeamercolor{block title}{fg=MidnightBlue,bg=white}
  \setbeamercolor{block body}{fg=black, bg=gray!10}
}
\newtheorem*{remark_fr*}{Remarque}
\BeforeBeginEnvironment{remark_fr*}{
  \setbeamerfont{block title}{series=\bfseries, shape=\itshape}
  \setbeamercolor{block title}{fg=MidnightBlue,bg=white}
  \setbeamercolor{block body}{fg=black, bg=gray!10}
}








\usepackage[svgnames]{xcolor}
\usepackage{tikz}
\usetikzlibrary{shapes.geometric, arrows}
\usepackage{enumerate}   
\usepackage{multirow}
\usepackage{txfonts}
\usepackage{mathrsfs}
\usepackage{pgfplots}
\pgfplotsset{compat = newest}
\usetikzlibrary{positioning, arrows.meta}
\usepgfplotslibrary{fillbetween}
\newcommand{\A}{(0,0) ++(135:2) circle (2)}
\newcommand{\B}{(0,0) ++(45:2) circle (2)}
\DeclareMathOperator{\C}{C}
\DeclareMathOperator{\util}{u}
%\setbeamersize{text margin left=1.5em,text margin right=1.5em} 
%\setbeamersize{text margin left=1.2cm,text margin right=1.2cm} 
\setbeamersize{text margin left=1.5em,text margin right=1.5em} 
%\usepackage{xr}
%\externaldocument{Econometrie1_UGA_P2e}
  \usepackage{eso-pic}
%\newcommand\AtPagemyUpperLeft[1]{\AtPageLowerLeft{%
%\put(\LenToUnit{0.9\paperwidth},\LenToUnit{0.85\paperheight}){#1}}}
%\AddToShipoutPictureFG{
 % \AtPagemyUpperLeft{{\includegraphics[width=1.1cm,keepaspectratio]{logoUGA2020.pdf}}}
%}%

%\setbeamercolor{title}{fg=black}
%\setbeamercolor{frametitle}{fg=black}
%\setbeamercolor{section in head/foot}{fg=black}
%\setbeamercolor{author in head/foot}{bg=Brown}
%\setbeamercolor{date in head/foot}{fg=Brown}
\AtBeginSection[]
  {
    \ifnum \value{framenumber}>1
      \begin{frame}<beamer>
      \frametitle{Plan}
      \tableofcontents[currentsection]
      \end{frame}
    \else
    \fi
  }
\setbeamertemplate{section page}
{
    \begin{centering}
    \begin{beamercolorbox}[sep=11pt,center]{part title}
    \usebeamerfont{section title}\thesection.~\insertsection\par
    \end{beamercolorbox}
    \end{centering}
}

%\titlegraphic{\includegraphics[width=1cm]{logoUGA2020.pdf}}
\title[]{ \textbf{Inférence statistique}}
\subtitle{Cours 5: Tests d'Hypothèses Paramétriques}
\date{\today}
\author{Michal W. Urdanivia\inst{*}}
\institute{\inst{*}UGA, Facult\'e d'\'Economie, GAEL, \\e-mail: \href{mailto:michal.wong-urdanivia@univ-grenoble-alpes.fr}{michal.wong-urdanivia@univ-grenoble-alpes.fr}}

%\titlegraphic{\includegraphics[width=1cm]{logoUGA2020.pdf}
%}
\begin{document}
%%% TIKZ STUFF
\usetikzlibrary{positioning}
\usetikzlibrary{snakes}
\usetikzlibrary{calc}
\usetikzlibrary{arrows}
\usetikzlibrary{decorations.markings}
\usetikzlibrary{shapes.misc}
\usetikzlibrary{matrix,shapes,arrows,fit,tikzmark}
\usetikzlibrary{shapes}
\tikzset{   
        every picture/.style={remember picture,baseline},
        every node/.style={anchor=base,align=center,outer sep=1.5pt},
        every path/.style={thick},
        }
\newcommand\marktopleft[1]{%
    \tikz[overlay,remember picture] 
        \node (marker-#1-a) at (-.3em,.3em) {};%
}
\newcommand\markbottomright[2]{%
    \tikz[overlay,remember picture] 
        \node (marker-#1-b) at (0em,0em) {};%
}
\tikzstyle{every picture}+=[remember picture] 
\tikzstyle{mybox} =[draw=black, very thick, rectangle, inner sep=10pt, inner ysep=20pt]
\tikzstyle{fancytitle} =[draw=black,fill=red, text=white]
\tikzstyle{observed}=[draw,circle,fill=gray!50]



\begin{frame}
\titlepage
\end{frame}
\begin{frame}{Plan}
 \tableofcontents
    \end{frame}
%\begin{frame}
%\frametitle{Contenu}
%\tableofcontents[pausesections, pausesubsections]
%\end{frame}

%\section{Qu'est-ce que l’économétrie ? A quoi (à qui) ça sert ?}
%\frame{\sectionpage}
%\begin{frame}
%  \tableofcontents  
%\end{frame}

\section{Introduction}
\frame{\sectionpage}

\begin{frame}[allowframebreaks]{Heuristique}
    \begin{itemize}
        \item \textbf{\underline{Idée clé:} décider de rejeter ou d'accepter des hypothèses.}
        \item \textbf{Exemples:}
        \begin{enumerate}[-]
            \item Après avoir lancé une pièce plusieurs fois, on veut décider(tester) à partir de ces résultats si 
            la pièce peut être considérée comme équilibrée.
            \item Après avoir prélevé le sang d'un patients et mesuré la concentration d'anticorps dans l'échantillon on veut décider(tester)
            si le patient peut être considéré comme porteur d'un virus.
        \end{enumerate}
        \item Formellement ceci peut être écrit ainsi:
        \begin{enumerate}[-]
            \item décider si $p = 1/2$ à partir d'un échantillon de $n$ v.a. de Bernouilli i.i.d.;
            \item décider si $c> c_0$ à partir d'un échantillon de $n$ v.a. i.i.d. de loi normale $\mathcal{N}(c, \sigma^2)$.
        \end{enumerate}
        \framebreak
        \item \textbf{\underline{Exemple 1:}} une pièce est lancée 80 fois, et le résultat est pile 54 fois. 
        Peut-on conclure \emph{significativement} que la pièce est déséquilibrée? \begin{enumerate}[-]
            \item $n = 80$, $X_1, X_2, \ldots, X_n \overset{i.i.d.}{\sim}\mathcal{B}er(p)$,
            \item $\bar{X}_n = 54/80 = 0.68$,
            \item Si cela était vrai que $p = 0.5$, par le TCL et le théorème de Slutsky,\begin{align*}
                \sqrt{n}\frac{\bar{X}_n - 0.5}{\sqrt{\bar{X}_n}(1-\bar{X}_n)} &\approx\mathcal{N}(0, 1),
            \end{align*}
            \item $\sqrt{n}\frac{\bar{X}_n - 0.5}{\sqrt{\bar{X}_n}(1-\bar{X}_n)} \approx 3.45$,
            \item conclusion: il paraît assez raisonnable de rejeter l'hypothèse "$p = 0.5$".
        \end{enumerate}
        \framebreak
        \item \textbf{\underline{Exemple 2:}} une pièce est lancée 30 fois, et le résultat est pile 13 fois. 
        Peut-on conclure \emph{significativement} que la pièce est déséquilibrée? \begin{enumerate}[-]
            \item $n = 80$, $X_1, X_2, \ldots, X_n \overset{i.i.d.}{\sim}\mathcal{B}er(p)$,
             \item $\bar{X}_n = 54/80 = 0.68$,
            \item Si cela était vrai que $p = 0.5$, par le TCL et le théorème de Slutsky,\begin{align*}
            \sqrt{n}\frac{\bar{X}_n - 0.5}{\sqrt{\bar{X}_n}(1-\bar{X}_n)} &\approx\mathcal{N}(0, 1),
             \end{align*}
             \item $\sqrt{n}\frac{\bar{X}_n - 0.5}{\sqrt{\bar{X}_n}(1-\bar{X}_n)} \approx -0.77$,
             \item conclusion: il paraît impossible de rejeter l'hypothèse "$p = 0.5$" de manière significative.
            \end{enumerate}
    \end{itemize}
\end{frame}


\section{Formulation Statistique}
\frame{\sectionpage}
\begin{frame}
    [allowframebreaks]{Définition du problème}
    \begin{itemize}
        \item Soit un échantillon de $n$ v.a. i.i.d. $X_1, X_2, \ldots, X_n$ et un modèle statistique 
        $\left(\mathcal{E}, \mathcal{F}, (\Prob_\theta)_{\theta\in \Theta}\right)$.
        \item Soit $\Theta_0$ et $\Theta_1$ des sous-ensembles disjoints de $\Theta$.
        \item Considérons deux hypothèses: $\left\{  \begin{array}{lll}
            H_0 &:&\theta\in \Theta_0\\
            H_1 &:&\theta\in \Theta_1
        \end{array}\right.$
        \item $H_0$ est l'\textbf{hypothèse nulle} et $H_1$ est l'\textbf{hypothèse alternative}.
        \item Si nous pensons que le vrai paramètre $\theta$ est soit dans $\Theta_0$, soit dans $\Theta_1$, alors on pourrait tester $H_0$ vs. $H_1$.
        \item On veut décider si oui ou non il faut rejeter $H_0$(on cherche de l'information dans les données à l'encontre de $H_0$).
    \framebreak
         \item $H_0$ et $H_1$ ne sont pas symétriques. Par exemple si $H_0$: "le patient est malade"($c>c_0$) vs. $H_1$: "le 
         patient est en bonne santé"($c\leq c_0$).
         \item Un \textbf{test} est une statistique $\delta\in\{0, 1\}$ telle que: 
         \begin{enumerate}[-]
             \item  Si $\delta = 0$, $H_0$ n'est pas rejetée;
             \item Si $\delta = 1$, $H_0$ est rejetée.
         \end{enumerate}
         \item Dans l'exemple du lancer de pièce: 
         \begin{enumerate}[-]
               \item $H_0:$ "$p=1/2$" vs. $H_1$: "$p\neq 1/2$",
               \item et $\delta = \Ind_{\abs{ \sqrt{n}\frac{\bar{X}_n - 0.5}{\sqrt{\bar{X}_n}(1-\bar{X}_n)}} > C}$ pour un seuil $C >0$.
               \item \textbf{Question:} comment choisir $C$?
         \end{enumerate}
         \framebreak
         \item \textbf{Région de rejet} d'un test $\delta$: \begin{align*}
             \mathcal{R}_{\delta} &=\{x\in \mathcal{E}^n: \delta(x) = 1\}.
         \end{align*}
         \item \textbf{Érreur du(de) premier(ère) type(espèce)} d'un test $\delta$(rejeter $H_0$ quand cette hypothèse est vraie):
         \begin{align*}
             \alpha_\delta : \Theta_0 &\rightarrow \R\\
             \theta &\mapsto \Prob_\theta(\delta = 1).
         \end{align*}
         \item \textbf{Érreur du(de) deuxième type(espèce)} d'un test $\delta$(ne pas rejeter $H_0$ quand $H_1$ vraie):
         \begin{align*}
            \beta_\delta : \Theta_1 &\rightarrow \R\\
            \theta &\mapsto \Prob_\theta(\delta = 0).
        \end{align*}
        \item \textbf{Puissance} du test $\delta$:
        \begin{align*}
            \pi_\delta &=\inf_{\theta\in\Theta_1}\left(1- \beta_\delta(\theta)\right).
        \end{align*}
        \framebreak 
        \item Le \textbf{niveau} d'un test $\delta$ est $\alpha$ si,\begin{align*}
            \alpha_\delta(\theta) &\leq \alpha,  \quad \forall \theta \in \Theta_0.
        \end{align*}
        \item Le \textbf{niveau asymptotique} d'un test $\delta$ est $\alpha$ si,\begin{align*}
            \lim_{n\to \infty}\alpha_\delta(\theta) &\leq \alpha,  \quad \forall \theta \in \Theta_0.
        \end{align*}
        \item Typiquement la forme d'un test est donnée par,
        \begin{align*}
            \delta &=\Ind_{T_n> c},
        \end{align*}
        pour une statistique $T_n$ et un seuil $c\in \R$.
        \item $T_n$ est alors appelée \textbf{statistique de test} et la région $\mathcal{R}_\delta$ est, 
        \begin{align*}
            \mathcal{R}_\delta := \left\{T_n > c\right\}.
        \end{align*}
    \end{itemize}
\end{frame}
\begin{frame}
    [allowframebreaks]{Un exemple}
    \begin{itemize}
        \item $X_1, X_2, \ldots, X_n \overset{i.i.d.}{\sim} \mathcal{B}er(p)$, pour $p\in (0, 1)$.
        \item Test: \begin{align*}H_0: "p=1/2" &\text{vs.} H_1:  "p\neq 1/2",\end{align*}
        avec un niveau asymptotique $\alpha\in(0,1)$.
        \item Statistique: \begin{align*}
            T_n &= \abs{ \sqrt{n}\frac{\hat{p}_n - 0.5}{\sqrt{\hat{p}_n(1-\hat{p}_n)}}}, 
        \end{align*}
        où $\hat{p}_n$ est l'estimateur du MV.
        \item Si $H_0$ est vraie, alors par le TCL et le théorème de Slutsky, \begin{align*}
            \Prob\left(T_n > q_{1-\alpha/2}\right)&\underset{n\to\infty}{\rightarrow} 0.05.
        \end{align*}
        \item Soit $\delta_\alpha = \Ind_{T_n>q_{1-\alpha/2}}$.
        
        \framebreak 

        \item Revenons cas du lancer de pièce et fixons $\alpha=5\%$, d'où $q_{1-\alpha/2} = 1.96$, et alors:
        \begin{enumerate}[-]
            \item Dans l'exemple 1, $H_0$ est rejeté au niveau asymptotique de $5\%$ par le test $\delta_{5\%}$.
            \item Dans l'exemple 2, $H_0$ n'est pas rejeté au niveau asymptotique de $5\%$ par le test $\delta_{5\%}$
        \end{enumerate}
        \item \textbf{Question}: dans l'exemple 1, pour quel niveau $\alpha$, le test $\delta_\alpha$ ne rejetterait $H_0$? et dans 
        l'exemple 2 pour quel niveau $\alpha$ le test $\delta_\alpha$ rejetterait $H_0$?
    \end{itemize}
\end{frame}
\begin{frame}
    [allowframebreaks]{P-value}
    \begin{itemize}
        \item \textbf{\underline{Définition}}: la p-value(asymptotique) d'un test $\delta_\alpha$ est le plus petit niveau(asymptotique)
         $\alpha$ pour lequel $\delta_\alpha$ conduit à rejeter $H_0$. Elle est aléatoire car elle dépend de l'échantillon.
         \item \textbf{\underline{Régle d'or:}} \begin{enumerate}[-]
             \item $\text{p-value} < \alpha$ $\Leftrightarrow H_0$ est rejeté par $\delta_\alpha$ pour le niveau(asymptotique)$\alpha$.
             \item Plus la p-value est petite, et plus on est confiant dans le rejet de $H_0$.
         \end{enumerate}
         \item Exemple 1: $\text{p-value} =\Prob\left(\abs{Z}>3.45\right)\ll 0.01$. 
         \item Exemple 2: $\text{p-value} =\Prob\left(\abs{Z}>0.77\right)\approx 0.44$. 
    \end{itemize}
\end{frame}
\section{Paradigme de Neyman-Pearson}
\frame{\sectionpage}
\begin{frame}
    [allowframebreaks]{Idée de Neyman-Pearson}
    \begin{itemize}
        \item Pour une hypothèse donnée, parmi tous les tests de niveau(ou de niveau asymptotique)
        $\alpha$, est-il possible de trouver celui qui a la puissance maximale?
        \item \textbf{Exemple:} le test trivial $\delta = 0$ qui ne rejette jamais $H_0$ a un niveau parfait($\alpha = 0$) 
        mais une puissance médiocre($\pi_\delta = 0$).
        \item La théorie de Neyman-Pearson offre des test qui sont(le plus) puissant pour un niveau donné.
        \item Nous allons seulement considérer certains cas.
    \end{itemize}
\end{frame}
\begin{frame}
    [allowframebreaks]{Rappel sur la loi du $\chi^2$}
    \begin{itemize}
        \item \textbf{\underline{Définition:}} pour un entier positif $d$, la loi du $\chi^2$ à $d$ degrés de liberté est la loi 
        de la v.a. $Z_1^2 + Z_2^2+\dots+ Z_d^2$, où $Z_1, Z_2, \ldots, Z_d\overset{i.i.d.}{\sim}\mathcal{N}(0,1)$.
        \item Exemples:\begin{enumerate}[-]
            \item Si $Z\sim\mathcal{N}_d(\boldzero, \Id_d)$, alors $\norm{Z}^2_2 \sim \chi_d^2$.
            \item Le théorème de Cochran énonce que pour $X_1, X_2, \ldots, X_n \overset{i.i.d.}{\sim} 
            \mathcal{N}(\mu, \sigma^2)$, si $S_n$ est la variance empirique, alors,\begin{align*}
                \frac{nS_n}{\sigma^2}&\sim \chi^2_{n-1}.
            \end{align*}
            \item $\chi^2_2 = \mathcal{E}xp(1/2)$.
        \end{enumerate}
    \end{itemize}
\end{frame}
\begin{frame}
    [allowframebreaks]{Rappel sur la loi de Student}
    \begin{itemize}
        \item \textbf{\underline{Définition:}} pour un entier positif $d$, la loi de Student à $d$ degrés de liberté 
        (notée $t_d$) est la loi de la v.a. $\frac{U}{\sqrt{V/d}}$, où $U\sim\mathcal{N}(0,1)$, $V\sim\chi^2_d$, et $U\pperp{V}$.
         \item Exemple: le théorème de Cochran énonce que pour $X_1, X_2, \ldots, X_n \overset{i.i.d.}{\sim}
         \mathcal{N}(\mu, \sigma^2)$, si $S_n$ est la variance empirique, alors,\begin{align*}
         \sqrt{n-1}\left(\frac{\bar{X}_n-\mu}{\sqrt{S}_n}\right)&\sim t_{n-1}.
          \end{align*}
    \end{itemize}
\end{frame}

\begin{frame}
    [allowframebreaks]{Test de Wald}
    \begin{itemize}
        \item Soit un échantillon de v.a. i.i.d., $X1, X_2,\ldots, X_n$ avec le modèle statistique 
        $\left(\mathbf{E}, \mathbf{F}, (\Prob_{\theta})_{\theta\in\Theta}\right)$, où $\Theta\subseteq \R^d$(pour $d\geq 1$), et soit $\theta_0\in\Theta$ une valeur donnée de $\theta$. 
        \item Considérons les hypothèses suivantes: \[\left\{
            \begin{array}{lll}
                H_0:&\theta  &= \theta_0,\\
                H_1:&\theta  &\neq \theta_0.
            \end{array}
        \right.\]
            \item Soit $\hat{\theta}^{MV}$ l'estimateur du MV, et supposons que les conditions techniques pour cet estimateur soient 
            satisfaites.
            \item Si $H_0$ est vraie, alors,\begin{align*}
                \sqrt{n} \mathcal{I}(\hat{\theta}^{MV})^{1/2}\left(\hat{\theta}_n^{MV}-\theta_0\right) 
                &\limd \mathcal{N}_d(\boldzero, \Id_d), \quad \text{par rapport à }  \Prob_{\theta_0}.
            \end{align*}
        \framebreak
        \item D'où, \begin{align*}
            \underbrace{n\left(\hat{\theta}_n^{MV}-\theta_0\right)^\top\mathcal{I}(\hat{\theta}^{MV})\left(\hat{\theta}_n^{MV}-\theta_0\right)}_{T_n}
            &\limd \chi_d^2, \quad \text{par rapport à }  \Prob_{\theta_0}.
        \end{align*}
        \item Test de Wald de niveau asymptotique $\alpha\in(0,1)$:\begin{align*}
            \delta &=\Ind_{T_n>q_{1-\alpha}},
        \end{align*}
        où $q_{1-\alpha}$ est le quantile $(1-\alpha)$ du $\chi^2_d$(voir tables).
        \item Remarque: le test de Wald est aussi valide si $H_1$ présente la forme "$\theta > \theta_0$" ou 
        "$\theta < \theta_0$" où "$\theta = \theta_1$", \ldots.
    \end{itemize}
\end{frame}
\begin{frame}
    [allowframebreaks]{Test du rapport de vraisemblance}
    \begin{itemize}
        \item Soit un échantillon de v.a. i.i.d., $X1, X_2,\ldots, X_n$ avec le modèle statistique 
         $\left(\mathbf{E}, \mathbf{F}, (\Prob_{\theta})_{\theta\in\Theta}\right)$, où 
         $\Theta\subseteq \R^d$(pour $d\geq 1$).
         \item On considère que l'hypothèse nulle présente la forme, \begin{align*}
             H_0 : (\theta_{r+1}, \theta_{r+2}, \ldots, \theta_d) &= (\theta_{r+1}^0, \theta_{r+2}^0, \ldots, \theta_d^0),
         \end{align*}
         pour des valeurs données $\theta_{r+1}^0, \theta_{r+2}^0, \ldots, \theta_d^0$.
         \item Soit,
         \begin{align*}
             \hat{\theta}_n &=\arg\max_{\theta\in\Theta}\ell_n(\theta), \quad\text{(estimateur du MV)},
         \end{align*}
         et,
         \begin{align*}
            \hat{\theta}_n^c &=\arg\max_{\theta\in\Theta_0}\ell_n(\theta), \quad\text{(estimateur du MV contraint)}.
         \end{align*}
         \framebreak 
        \item Statistique de test: \begin{align*}
            T_n &= 2\left(\ell_n(\hat{\theta}_n) - \ell_n(\hat{\theta}_n^c)\right).
        \end{align*}
        \item \textbf{\underline{Théorème:}} supposons que $H_0$ soit vraie et que les conditions 
        techniques de l'estimateur du MV soient satisfaites. Alors,
        \begin{align*}
            T_n &\limd \chi^2_{d-r}, \quad \text{par rapport à }  \Prob_{\theta}.
        \end{align*}
        \item Test du rapport de vraisemblance de niveau asymptotique $\alpha\in(0, 1)$: 
        \begin{align*}
            \delta &=\Ind_{T_n > q_{1-\alpha}},
        \end{align*}
        où $q_{1-\alpha}$ est le quantile $(1-\alpha)$ du $\chi^2_{d-r}$(voir tables).
    \end{itemize}
\end{frame}
\begin{frame}
    [allowframebreaks]{Tests d'hypothèses implicites}
    \begin{itemize}
        \item Soit $X_1, X_2, \ldots, X_n$ des v.a. i.i.d., et soit $\theta\in\R^d$ un paramètre associé à la loi de $X$(une v.a. "générique"). 
        Cela peut être par exemple, un moment, un paramètre du modèle statistique, etc.
        \item Soit $g:\R^d\to\R^k$ une application continûment dérivable(avec $k<d$).
        \item Considérons les hypothèses suivantes:
        \[
        \begin{array}{lll}
            H_0:&g(\theta)  &= 0\\
            H_1:&g(\theta)  &\neq 0.
        \end{array}\]
        \item Par exemple: $g(\theta) = (\theta_1, \theta_2)$(ici $k=2$), ou $g(\theta) = \theta_1-\theta_2$(ici $k=1$), \ldots
    
    \framebreak 
    \item Supposons qu'on dispose d'un estimateur asymptotiquement normal $\hat{\theta}_n$: \begin{align*}
        \sqrt{n}(\hat{\theta}_n  - \theta)&\limd \mathcal{N}_d\left(\boldzero, \Sigma(\theta)\right).
    \end{align*}
    \item Méthode Delta: \begin{align*}
        \sqrt{n}\left(g(\hat{\theta}_n) - g(\theta)\right)&\limd \mathcal{N}_k(\boldzero, \Gamma(\theta)),
    \end{align*}
    où $\Gamma(\theta) = \nabla g(\theta)^\top \Sigma(\theta)\nabla g(\theta)\in\R^{k\times k}$.
    \item Supposons que $\Sigma(\theta)$ soit inversible, et $\nabla g(\theta)$ de rang $k$, de sorte que $\Gamma(\theta)$ 
    soit aussi inversible et que, 
    \begin{align*}
        \sqrt{n} \Gamma(\theta)^{-1/2}\left(g(\hat{\theta}_n) - g(\theta)\right)&\limd \mathcal{N}_k(\boldzero, \Id_k).
    \end{align*}
    \framebreak
    \item Par le théorème de Slutsky si $\Gamma(\theta)$ est continue en $\theta$,
    \begin{align*}
        \sqrt{n} \Gamma(\hat{\theta}_n)^{-1/2}\left(g(\hat{\theta}_n )- g(\theta)\right)&\limd \mathcal{N}_k(\boldzero, \Id_k).
    \end{align*}
    \item En conséquence, si $H_0$ est vraie, i.e., $g(\theta) = 0$,
    \begin{align*}
        \underbrace{n g(\hat{\theta}_n)^\top \Gamma^{-1}(\hat{\theta}_n)g(\hat{\theta}_n)}_{T_n} &\limd \chi^2_k.
    \end{align*}
    \item Test de niveau asymptotique $\alpha\in(0, 1)$: 
    \begin{align*}
        \delta &=\Ind_{T_n > q_{1-\alpha}},
    \end{align*}
    où $q_{1-\alpha}$ est le quantile $(1-\alpha)$ du $\chi^2_k$(voir tables).
    \end{itemize}
\end{frame}
\begin{frame}
    [allowframebreaks]{Le cas multinomial: test du $\chi^2$}
    \begin{itemize}
        \item Soit $\mathcal{E} = \{a_1, a_2, \ldots, a_k\}$ un ensemble dénombrable et 
        $(\Prob_{\mathbf{p}})_{\mathbf{p}\in\Delta_k}$ la famille de toutes les lois de probabilité sur $\mathcal{E}$: 
        \begin{align*}
            \Delta_k &=\{\mathbf{p} = (\prob_1, \prob_2, \ldots, \prob_k)\in(0, 1)^K: 
            \sum_{j=1}^k\prob_j = 1\}.
        \end{align*}
        \item Pour $\mathbf{p}\in\Delta_k$ et $X\sim \Prob_{\mathbf{p}}$,
        \begin{align*}
            \Prob_{\mathbf{p}}(X=a_j) &= \prob_j, \quad j = 1,2, \ldots, k.
        \end{align*}

        \framebreak
        
        \item Soit $X_1, X_2, \ldots, X_n\overset{i.i.d.}{\sim} \Prob_{\mathbf{p}}$, pour un $\mathbf{p}\in\Delta_k$ inconnu, 
        $\mathbf{p}^{0}\in\Delta^K$ donné.
        \item On souhaite tester: "$H_0: \mathbf{p} = \mathbf{p}^0$" vs., "$H_1: \mathbf{p} \neq \mathbf{p}^0$" 
        au niveau asymptotique $\alpha\in(0,1).$
        \item Exemple: si $\mathbf{p}^0 = (1/k, 1/k, \ldots, 1:k)$, on teste si $\Prob_{\mathbf{p}}$ est la loi uniforme sur $\mathcal{E}$.
         \framebreak
         \item Vraisemblance du modèle:
         \begin{align*}
             \mathcal{L}_n(X_1, X_2, \ldots, X_n, \boldp) &= \prob_1^{N_1}\prob_{2}^{N_2}\ldots\prob_k^{N_k},
         \end{align*}
         où $N_j = \#\left\{i = 1, 2, \ldots, n: X_i=a_j\right\}$.
         \item Soit $\hat{\boldp}$ l'estimateur du MV: \begin{align*}
            \hat{\prob} &= \frac{N_j}{n}, \quad j = 1, 2, \ldots, k.
         \end{align*}
         \item \textbf{Remarque:} $\hat{\boldp}$ maximise $\mathcal{L}_n(X_1, X_2, \ldots, X_n, \boldp)$ 
         \textbf{sous la contrainte}, \begin{align*}
             \sum_{j=1}^k\prob_j &=1.
         \end{align*}
         \framebreak 
         \item Si $H_0$ est vraie, alors $\sqrt{n}(\hat{\boldp} - \boldp^0)$ est asymptotiquement normale, 
          et le résultat suivant est établi,
          \item \textbf{\underline{Théorème:}} \begin{align*}
              \underbrace{n\sum_{j=1}^k\frac{(\hat{\prob}_j - \prob_j^0)^2}{\prob_j^0}}_{T_n}&\limd \chi^2_{k-1}.
          \end{align*}
          \item Test du $\chi^2$ de niveau asymptotique $\alpha$ : $\delta_\alpha = \Ind_{T_n>q_{1-\alpha}}$, où $q_{1-\alpha}$ est 
          quantile d'ordre $1-\alpha$ du $\chi^2_{k-1}$.
          \item La p-value asymptotique de ce test est: $\prob-\text{value} = \Prob(Z> T_n| T_n)$, où $Z\sim\chi^2_{k-1}$ 
          et $Z\pperp T_n$.
    \end{itemize}
\end{frame}

\begin{frame}
    [allowframebreaks]{Cas Gaussien: test de Student}
    \begin{itemize}
        \item Soit $X_1, X_2, \ldots, X_n\simiid\mathcal{N}(\mu, \sigma^2)$ pour $\mu\in\R$, $\sigma^2>0$, 
         et soit aussi un $\mu_0\in \R$  donné.
         \item On veut tester: \begin{align*}
             H_0: "\mu = \mu_0" &\text{vs.} "H_1 : \mu\neq \mu_0,
         \end{align*}
         au niveau asymptotique $\alpha\in(0,1)$.
         \item \textbf{1er cas}: $\sigma^2$ est connu.
         \begin{enumerate}[-]
             \item Soit $T_n  = \sqrt{n}\left(\frac{\bar{X}_n - \mu_0}{\sigma}\right)$. 
             \item Alors $T_n\sim \mathcal{N}(0,1)$ et, 
             \begin{align*}
                 \delta_\alpha = \Ind_{\abs{T_n} > q_{1-\alpha / 2}}
             \end{align*}
             est un test au niveau(non asymptotique) $\alpha$.
         \end{enumerate}
         \framebreak
         \item \textbf{2ème cas}: $\sigma^2$ est inconnu.
         \begin{enumerate}[-]
            \item Soit $\tilde{T}_n  = \sqrt{n - 1}\left(\frac{\bar{X}_n - \mu_0}{\sqrt{S_n}}\right)$, 
            où $S_n$ est la variance empirique.
            \item Théorème de Cochran : \begin{enumerate}[-]
                \item $\bar{X}_n \pperp S_n$;
                \item $\frac{nS_n}{\sigma^2}\sim\chi^2_{n-1}$.
            \end{enumerate}
            \item Et par conséquent, $\tilde{T}_n \sim t_{n-1}$: loi de Student à $n-1$ degrés de liberté.
            \framebreak
            \item Test de Student de niveau(non asymptotique) $\alpha\in(0,1)$: 
            \begin{align*}
                \delta_\alpha = \Ind_{\abs{\tilde{T}_n} > q_{1-\alpha / 2}},
            \end{align*}
            où $q_{1-\alpha / 2}$ est le quantile d'ordre $1-\alpha/2$ de $t_{n-1}$.
            \item  Si $H_1$ est $"\mu>\mu_0"$, le test de Student de niveau $\alpha\in(0,1)$ est:
            \begin{align*}
                \delta^\prime_\alpha = \Ind_{\tilde{T}_n > q_{1-\alpha}},
            \end{align*}
            où $q_{1-\alpha}$ est le quantile d'ordre $1-\alpha$ de $t_{n-1}$.
        \end{enumerate}
        \item Avantage du test de Student: 
        \begin{enumerate}[-]
            \item Non asymptotique.
            \item Peut être réalisé sur des échantillons petits.
        \end{enumerate}
        \item Inconvénients du test de Student: suppose que l'échantillon est gaussien.
    \end{itemize}
\end{frame}
\end{document}